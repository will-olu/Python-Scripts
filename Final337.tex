% Options for packages loaded elsewhere
\PassOptionsToPackage{unicode}{hyperref}
\PassOptionsToPackage{hyphens}{url}
%
\documentclass[
]{article}
\usepackage{lmodern}
\usepackage{amssymb,amsmath}
\usepackage{ifxetex,ifluatex}
\ifnum 0\ifxetex 1\fi\ifluatex 1\fi=0 % if pdftex
  \usepackage[T1]{fontenc}
  \usepackage[utf8]{inputenc}
  \usepackage{textcomp} % provide euro and other symbols
\else % if luatex or xetex
  \usepackage{unicode-math}
  \defaultfontfeatures{Scale=MatchLowercase}
  \defaultfontfeatures[\rmfamily]{Ligatures=TeX,Scale=1}
\fi
% Use upquote if available, for straight quotes in verbatim environments
\IfFileExists{upquote.sty}{\usepackage{upquote}}{}
\IfFileExists{microtype.sty}{% use microtype if available
  \usepackage[]{microtype}
  \UseMicrotypeSet[protrusion]{basicmath} % disable protrusion for tt fonts
}{}
\makeatletter
\@ifundefined{KOMAClassName}{% if non-KOMA class
  \IfFileExists{parskip.sty}{%
    \usepackage{parskip}
  }{% else
    \setlength{\parindent}{0pt}
    \setlength{\parskip}{6pt plus 2pt minus 1pt}}
}{% if KOMA class
  \KOMAoptions{parskip=half}}
\makeatother
\usepackage{xcolor}
\IfFileExists{xurl.sty}{\usepackage{xurl}}{} % add URL line breaks if available
\IfFileExists{bookmark.sty}{\usepackage{bookmark}}{\usepackage{hyperref}}
\hypersetup{
  pdftitle={Final},
  pdfauthor={Ade Olu-Ajeigbe},
  hidelinks,
  pdfcreator={LaTeX via pandoc}}
\urlstyle{same} % disable monospaced font for URLs
\usepackage[margin=1in]{geometry}
\usepackage{color}
\usepackage{fancyvrb}
\newcommand{\VerbBar}{|}
\newcommand{\VERB}{\Verb[commandchars=\\\{\}]}
\DefineVerbatimEnvironment{Highlighting}{Verbatim}{commandchars=\\\{\}}
% Add ',fontsize=\small' for more characters per line
\usepackage{framed}
\definecolor{shadecolor}{RGB}{248,248,248}
\newenvironment{Shaded}{\begin{snugshade}}{\end{snugshade}}
\newcommand{\AlertTok}[1]{\textcolor[rgb]{0.94,0.16,0.16}{#1}}
\newcommand{\AnnotationTok}[1]{\textcolor[rgb]{0.56,0.35,0.01}{\textbf{\textit{#1}}}}
\newcommand{\AttributeTok}[1]{\textcolor[rgb]{0.77,0.63,0.00}{#1}}
\newcommand{\BaseNTok}[1]{\textcolor[rgb]{0.00,0.00,0.81}{#1}}
\newcommand{\BuiltInTok}[1]{#1}
\newcommand{\CharTok}[1]{\textcolor[rgb]{0.31,0.60,0.02}{#1}}
\newcommand{\CommentTok}[1]{\textcolor[rgb]{0.56,0.35,0.01}{\textit{#1}}}
\newcommand{\CommentVarTok}[1]{\textcolor[rgb]{0.56,0.35,0.01}{\textbf{\textit{#1}}}}
\newcommand{\ConstantTok}[1]{\textcolor[rgb]{0.00,0.00,0.00}{#1}}
\newcommand{\ControlFlowTok}[1]{\textcolor[rgb]{0.13,0.29,0.53}{\textbf{#1}}}
\newcommand{\DataTypeTok}[1]{\textcolor[rgb]{0.13,0.29,0.53}{#1}}
\newcommand{\DecValTok}[1]{\textcolor[rgb]{0.00,0.00,0.81}{#1}}
\newcommand{\DocumentationTok}[1]{\textcolor[rgb]{0.56,0.35,0.01}{\textbf{\textit{#1}}}}
\newcommand{\ErrorTok}[1]{\textcolor[rgb]{0.64,0.00,0.00}{\textbf{#1}}}
\newcommand{\ExtensionTok}[1]{#1}
\newcommand{\FloatTok}[1]{\textcolor[rgb]{0.00,0.00,0.81}{#1}}
\newcommand{\FunctionTok}[1]{\textcolor[rgb]{0.00,0.00,0.00}{#1}}
\newcommand{\ImportTok}[1]{#1}
\newcommand{\InformationTok}[1]{\textcolor[rgb]{0.56,0.35,0.01}{\textbf{\textit{#1}}}}
\newcommand{\KeywordTok}[1]{\textcolor[rgb]{0.13,0.29,0.53}{\textbf{#1}}}
\newcommand{\NormalTok}[1]{#1}
\newcommand{\OperatorTok}[1]{\textcolor[rgb]{0.81,0.36,0.00}{\textbf{#1}}}
\newcommand{\OtherTok}[1]{\textcolor[rgb]{0.56,0.35,0.01}{#1}}
\newcommand{\PreprocessorTok}[1]{\textcolor[rgb]{0.56,0.35,0.01}{\textit{#1}}}
\newcommand{\RegionMarkerTok}[1]{#1}
\newcommand{\SpecialCharTok}[1]{\textcolor[rgb]{0.00,0.00,0.00}{#1}}
\newcommand{\SpecialStringTok}[1]{\textcolor[rgb]{0.31,0.60,0.02}{#1}}
\newcommand{\StringTok}[1]{\textcolor[rgb]{0.31,0.60,0.02}{#1}}
\newcommand{\VariableTok}[1]{\textcolor[rgb]{0.00,0.00,0.00}{#1}}
\newcommand{\VerbatimStringTok}[1]{\textcolor[rgb]{0.31,0.60,0.02}{#1}}
\newcommand{\WarningTok}[1]{\textcolor[rgb]{0.56,0.35,0.01}{\textbf{\textit{#1}}}}
\usepackage{graphicx,grffile}
\makeatletter
\def\maxwidth{\ifdim\Gin@nat@width>\linewidth\linewidth\else\Gin@nat@width\fi}
\def\maxheight{\ifdim\Gin@nat@height>\textheight\textheight\else\Gin@nat@height\fi}
\makeatother
% Scale images if necessary, so that they will not overflow the page
% margins by default, and it is still possible to overwrite the defaults
% using explicit options in \includegraphics[width, height, ...]{}
\setkeys{Gin}{width=\maxwidth,height=\maxheight,keepaspectratio}
% Set default figure placement to htbp
\makeatletter
\def\fps@figure{htbp}
\makeatother
\setlength{\emergencystretch}{3em} % prevent overfull lines
\providecommand{\tightlist}{%
  \setlength{\itemsep}{0pt}\setlength{\parskip}{0pt}}
\setcounter{secnumdepth}{-\maxdimen} % remove section numbering

\title{Final}
\author{Ade Olu-Ajeigbe}
\date{5/1/2021}

\begin{document}
\maketitle

\begin{Shaded}
\begin{Highlighting}[]
\KeywordTok{library}\NormalTok{(latexpdf)}
\end{Highlighting}
\end{Shaded}

\begin{verbatim}
## Warning: package 'latexpdf' was built under R version 4.0.3
\end{verbatim}

\begin{Shaded}
\begin{Highlighting}[]
\KeywordTok{library}\NormalTok{(readxl)}
\end{Highlighting}
\end{Shaded}

\begin{verbatim}
## Warning: package 'readxl' was built under R version 4.0.5
\end{verbatim}

\begin{Shaded}
\begin{Highlighting}[]
\NormalTok{data <-}\StringTok{ }\KeywordTok{read_excel}\NormalTok{(}\StringTok{"whitewines.xls"}\NormalTok{)}


\NormalTok{final<-data[}\KeywordTok{sample}\NormalTok{(}\KeywordTok{nrow}\NormalTok{(data), }\DecValTok{1500}\NormalTok{), ]}


\NormalTok{final}\OperatorTok{$}\NormalTok{quality <-}\StringTok{ }\KeywordTok{ifelse}\NormalTok{(final}\OperatorTok{$}\NormalTok{quality }\OperatorTok{>}\StringTok{ }\DecValTok{6}\NormalTok{, }\StringTok{"high"}\NormalTok{, }\KeywordTok{as.character}\NormalTok{(final}\OperatorTok{$}\NormalTok{quality))}

\NormalTok{final}\OperatorTok{$}\NormalTok{quality <-}\StringTok{ }\KeywordTok{ifelse}\NormalTok{(final}\OperatorTok{$}\NormalTok{quality }\OperatorTok{<}\StringTok{ }\DecValTok{5}\NormalTok{, }\StringTok{"low"}\NormalTok{, }\KeywordTok{as.character}\NormalTok{(final}\OperatorTok{$}\NormalTok{quality))}

\NormalTok{final}\OperatorTok{$}\NormalTok{quality <-}\StringTok{ }\NormalTok{final}\OperatorTok{$}\NormalTok{quality }\OperatorTok{==}\StringTok{ "high"} 

\KeywordTok{head}\NormalTok{(final)}
\end{Highlighting}
\end{Shaded}

\begin{verbatim}
## # A tibble: 6 x 12
##   `fixed acidity` `volatile acidity` `citric acid` `residual sugar` chlorides
##             <dbl>              <dbl>         <dbl>            <dbl>     <dbl>
## 1             6.5               0.25          0.5               7.6     0.047
## 2             7.4               0.16          0.27             15.5     0.05 
## 3             7.3               0.14          0.49              1.1     0.038
## 4             7.4               0.31          0.48             14.2     0.042
## 5             8.1               0.24          0.26             11       0.043
## 6             7.4               0.27          0.49              1.1     0.037
## # ... with 7 more variables: free sulfur dioxide <dbl>,
## #   total sulfur dioxide <dbl>, density <dbl>, pH <dbl>, sulphates <dbl>,
## #   alcohol <dbl>, quality <lgl>
\end{verbatim}

\begin{Shaded}
\begin{Highlighting}[]
\CommentTok{#PartA)}
\NormalTok{cols<-}\KeywordTok{c}\NormalTok{(}\DecValTok{1}\OperatorTok{:}\DecValTok{11}\NormalTok{) }\CommentTok{#columns numbers 1 through 11}
\NormalTok{final[cols]<-}\KeywordTok{sapply}\NormalTok{(final[cols], as.numeric)}\CommentTok{#changes first 11 columns to numeric}



\NormalTok{model<-}\KeywordTok{glm}\NormalTok{(quality }\OperatorTok{~}\StringTok{ }\NormalTok{.,}\DataTypeTok{family =} \KeywordTok{binomial}\NormalTok{(logit), }\DataTypeTok{data=}\NormalTok{final)}
\KeywordTok{summary}\NormalTok{(model)}
\end{Highlighting}
\end{Shaded}

\begin{verbatim}
## 
## Call:
## glm(formula = quality ~ ., family = binomial(logit), data = final)
## 
## Deviance Residuals: 
##     Min       1Q   Median       3Q      Max  
## -1.8967  -0.6767  -0.3945  -0.1782   2.6770  
## 
## Coefficients:
##                          Estimate Std. Error z value Pr(>|z|)    
## (Intercept)             9.002e+02  1.787e+02   5.039 4.68e-07 ***
## `fixed acidity`         7.948e-01  1.651e-01   4.814 1.48e-06 ***
## `volatile acidity`     -3.312e+00  8.557e-01  -3.870 0.000109 ***
## `citric acid`          -7.491e-01  7.432e-01  -1.008 0.313481    
## `residual sugar`        3.764e-01  6.676e-02   5.638 1.72e-08 ***
## chlorides              -7.142e+00  6.461e+00  -1.106 0.268935    
## `free sulfur dioxide`   9.214e-03  5.834e-03   1.579 0.114262    
## `total sulfur dioxide` -1.004e-03  2.797e-03  -0.359 0.719628    
## density                -9.251e+02  1.810e+02  -5.111 3.20e-07 ***
## pH                      3.688e+00  7.756e-01   4.754 1.99e-06 ***
## sulphates               2.314e+00  6.374e-01   3.631 0.000283 ***
## alcohol                -1.645e-01  2.155e-01  -0.763 0.445263    
## ---
## Signif. codes:  0 '***' 0.001 '**' 0.01 '*' 0.05 '.' 0.1 ' ' 1
## 
## (Dispersion parameter for binomial family taken to be 1)
## 
##     Null deviance: 1593.3  on 1499  degrees of freedom
## Residual deviance: 1263.7  on 1488  degrees of freedom
## AIC: 1287.7
## 
## Number of Fisher Scoring iterations: 6
\end{verbatim}

\begin{Shaded}
\begin{Highlighting}[]
\CommentTok{# residual sugar, volatile.acidity,density, pH, sulphates, and fixed.acidity are the significant values from least to greatest}

\CommentTok{#Part B)}
\CommentTok{#residual sugars were the most significant since they had the lowest p-values}

\KeywordTok{exp}\NormalTok{(model}\OperatorTok{$}\NormalTok{coefficients)}
\end{Highlighting}
\end{Shaded}

\begin{verbatim}
##            (Intercept)        `fixed acidity`     `volatile acidity` 
##                    Inf           2.213985e+00           3.645622e-02 
##          `citric acid`       `residual sugar`              chlorides 
##           4.727760e-01           1.456982e+00           7.908947e-04 
##  `free sulfur dioxide` `total sulfur dioxide`                density 
##           1.009257e+00           9.989964e-01           0.000000e+00 
##                     pH              sulphates                alcohol 
##           3.994585e+01           1.011878e+01           8.483386e-01
\end{verbatim}

\begin{Shaded}
\begin{Highlighting}[]
\CommentTok{#Part C)}
\NormalTok{model2<-}\KeywordTok{glm}\NormalTok{(quality }\OperatorTok{~}\StringTok{ `}\DataTypeTok{residual sugar}\StringTok{`} \OperatorTok{+}\StringTok{ `}\DataTypeTok{volatile acidity}\StringTok{`} \OperatorTok{+}\StringTok{ }\NormalTok{pH }\OperatorTok{+}\StringTok{ }\NormalTok{density }\OperatorTok{+}\StringTok{ }\NormalTok{sulphates }\OperatorTok{+}\StringTok{ `}\DataTypeTok{fixed acidity}\StringTok{`}\NormalTok{, }\DataTypeTok{family =} \KeywordTok{binomial}\NormalTok{(logit), }\DataTypeTok{data =}\NormalTok{final)}
\KeywordTok{summary}\NormalTok{(model2)}
\end{Highlighting}
\end{Shaded}

\begin{verbatim}
## 
## Call:
## glm(formula = quality ~ `residual sugar` + `volatile acidity` + 
##     pH + density + sulphates + `fixed acidity`, family = binomial(logit), 
##     data = final)
## 
## Deviance Residuals: 
##     Min       1Q   Median       3Q      Max  
## -1.8946  -0.6635  -0.3930  -0.1989   2.7140  
## 
## Coefficients:
##                      Estimate Std. Error z value Pr(>|z|)    
## (Intercept)         803.22860   55.56401  14.456  < 2e-16 ***
## `residual sugar`      0.34599    0.03242  10.673  < 2e-16 ***
## `volatile acidity`   -3.57235    0.81035  -4.408 1.04e-05 ***
## pH                    3.37970    0.57430   5.885 3.98e-09 ***
## density            -827.60976   57.21868 -14.464  < 2e-16 ***
## sulphates             2.19038    0.59455   3.684  0.00023 ***
## `fixed acidity`       0.69171    0.10946   6.319 2.63e-10 ***
## ---
## Signif. codes:  0 '***' 0.001 '**' 0.01 '*' 0.05 '.' 0.1 ' ' 1
## 
## (Dispersion parameter for binomial family taken to be 1)
## 
##     Null deviance: 1593.3  on 1499  degrees of freedom
## Residual deviance: 1270.2  on 1493  degrees of freedom
## AIC: 1284.2
## 
## Number of Fisher Scoring iterations: 5
\end{verbatim}

\begin{Shaded}
\begin{Highlighting}[]
\CommentTok{#Part D)}
\CommentTok{#The alcohol seems to not have an intense correlation between the two values present}

\CommentTok{#Part E)}
\CommentTok{#Linear regression cannot plot data well that is not inherently linear}
\end{Highlighting}
\end{Shaded}

\begin{enumerate}
\def\labelenumi{(\alph{enumi})}
\tightlist
\item
  (10 points) Perform k-fold cross validation with your model from 1(c)
  with k=10. Report the accuracy of your model.
\item
  (10 points) Report the accuracy of your model through leave one out
  cross validation. For this dataset, which method (k-fold or leave one
  out) would you recommend?
\end{enumerate}

\begin{Shaded}
\begin{Highlighting}[]
\NormalTok{x <-}\StringTok{ }\NormalTok{final[}\KeywordTok{sample}\NormalTok{(}\DecValTok{1}\OperatorTok{:}\KeywordTok{nrow}\NormalTok{(final)),]}
\KeywordTok{library}\NormalTok{(caret)}
\end{Highlighting}
\end{Shaded}

\begin{verbatim}
## Warning: package 'caret' was built under R version 4.0.5
\end{verbatim}

\begin{verbatim}
## Loading required package: lattice
\end{verbatim}

\begin{verbatim}
## Loading required package: ggplot2
\end{verbatim}

\begin{verbatim}
## Warning: package 'ggplot2' was built under R version 4.0.3
\end{verbatim}

\begin{Shaded}
\begin{Highlighting}[]
\KeywordTok{library}\NormalTok{(e1071)}
\end{Highlighting}
\end{Shaded}

\begin{verbatim}
## Warning: package 'e1071' was built under R version 4.0.5
\end{verbatim}

\begin{Shaded}
\begin{Highlighting}[]
\CommentTok{#set.seed(134)}
\NormalTok{Train <-}\StringTok{ }\KeywordTok{createDataPartition}\NormalTok{(final}\OperatorTok{$}\NormalTok{quality, }\DataTypeTok{p=}\FloatTok{0.75}\NormalTok{, }\DataTypeTok{list=}\OtherTok{FALSE}\NormalTok{)}
\NormalTok{training <-}\StringTok{ }\NormalTok{x[ Train, ]}
\NormalTok{testing <-}\StringTok{ }\NormalTok{x[ }\OperatorTok{-}\NormalTok{Train, ]}


\NormalTok{train.control <-}\StringTok{ }\KeywordTok{trainControl}\NormalTok{(}\DataTypeTok{method =} \StringTok{"repeatedcv"}\NormalTok{, }\DataTypeTok{number =} \DecValTok{10}\NormalTok{, }\DataTypeTok{savePredictions =} \OtherTok{TRUE}\NormalTok{, }\DataTypeTok{repeats =} \DecValTok{5}\NormalTok{)}


\NormalTok{training}\OperatorTok{$}\NormalTok{quality<-}\KeywordTok{factor}\NormalTok{(training}\OperatorTok{$}\NormalTok{quality)}



\NormalTok{kfoldtestss<-}\StringTok{ }\KeywordTok{train}\NormalTok{(quality }\OperatorTok{~}\StringTok{ `}\DataTypeTok{residual sugar}\StringTok{`} \OperatorTok{+}\StringTok{ `}\DataTypeTok{volatile acidity}\StringTok{`} \OperatorTok{+}\StringTok{ }\NormalTok{pH }\OperatorTok{+}\StringTok{ }\NormalTok{density }\OperatorTok{+}\StringTok{ }\NormalTok{sulphates }\OperatorTok{+}\StringTok{ `}\DataTypeTok{fixed acidity}\StringTok{`}\NormalTok{, }\DataTypeTok{data =}\NormalTok{ training, }\DataTypeTok{trControl =}\NormalTok{ train.control, }\DataTypeTok{method =} \StringTok{"glm"}\NormalTok{, }\DataTypeTok{family =} \StringTok{"binomial"}\NormalTok{)}
\KeywordTok{print}\NormalTok{(kfoldtestss)}
\end{Highlighting}
\end{Shaded}

\begin{verbatim}
## Generalized Linear Model 
## 
## 1126 samples
##    6 predictor
##    2 classes: 'FALSE', 'TRUE' 
## 
## No pre-processing
## Resampling: Cross-Validated (10 fold, repeated 5 times) 
## Summary of sample sizes: 1014, 1014, 1014, 1013, 1012, 1013, ... 
## Resampling results:
## 
##   Accuracy   Kappa    
##   0.8039173  0.3219497
\end{verbatim}

\begin{Shaded}
\begin{Highlighting}[]
\CommentTok{#Part B}

\NormalTok{acc <-}\StringTok{ }\OtherTok{NULL}
\ControlFlowTok{for}\NormalTok{(i }\ControlFlowTok{in} \DecValTok{1}\OperatorTok{:}\KeywordTok{nrow}\NormalTok{(final))}
\NormalTok{\{}
\NormalTok{train <-}\StringTok{ }\NormalTok{final[}\OperatorTok{-}\NormalTok{i,]}
\NormalTok{test <-}\StringTok{ }\NormalTok{final[i,]}
\CommentTok{# Fitting}
\NormalTok{loo_m <-}\StringTok{ }\KeywordTok{glm}\NormalTok{(quality }\OperatorTok{~}\StringTok{ `}\DataTypeTok{residual sugar}\StringTok{`} \OperatorTok{+}\StringTok{ `}\DataTypeTok{volatile acidity}\StringTok{`} \OperatorTok{+}\StringTok{ }\NormalTok{pH }\OperatorTok{+}\StringTok{ }\NormalTok{density }\OperatorTok{+}\StringTok{ }\NormalTok{sulphates }\OperatorTok{+}\StringTok{ `}\DataTypeTok{fixed acidity}\StringTok{`}\NormalTok{, }\DataTypeTok{family =}\KeywordTok{binomial}\NormalTok{(logit), }\DataTypeTok{data =}\NormalTok{ train)}
\CommentTok{# Predict results}
\NormalTok{pred <-}\StringTok{ }\KeywordTok{predict}\NormalTok{(loo_m,}\DataTypeTok{newdata=}\NormalTok{test,}\DataTypeTok{type=}\StringTok{"response"}\NormalTok{)}
\CommentTok{# If prob > 0.5 then 1, else 0}
\NormalTok{results <-}\StringTok{ }\KeywordTok{ifelse}\NormalTok{(pred }\OperatorTok{>}\StringTok{ }\FloatTok{0.5}\NormalTok{,}\DecValTok{1}\NormalTok{,}\DecValTok{0}\NormalTok{)}
\CommentTok{# Actual answers}
\NormalTok{answers <-}\StringTok{ }\NormalTok{test}\OperatorTok{$}\NormalTok{quality}
\CommentTok{# Calculate accuracy}
\NormalTok{misClasificError <-}\StringTok{ }\KeywordTok{mean}\NormalTok{(answers }\OperatorTok{!=}\StringTok{ }\NormalTok{results)}
\CommentTok{# Collecting results}
\NormalTok{acc[i] <-}\StringTok{ }\DecValTok{1}\OperatorTok{-}\NormalTok{misClasificError}
\NormalTok{\}}
\KeywordTok{mean}\NormalTok{(acc)}
\end{Highlighting}
\end{Shaded}

\begin{verbatim}
## [1] 0.8033333
\end{verbatim}

\begin{enumerate}
\def\labelenumi{\arabic{enumi}.}
\setcounter{enumi}{2}
\tightlist
\item
  Swarnali can have 0, 1 or 2 ice-creams in a day depending on whether
  it is a Hot or Cold day. Consider the following Hidden Markov Model:
  P(0\textbar H) = 0.2 P(1\textbar H) = 0.5 P(2\textbar H) = 0.3
  P(0\textbar C) = 0.6 P(1\textbar C) = 0.3 P(2\textbar C) = 0.1
\end{enumerate}

hot to cold = 0.3 cold to hot = 0.4 hot to hot = 0.7 cold to cold = 0.6

\begin{enumerate}
\def\labelenumi{(\alph{enumi})}
\tightlist
\item
  (10 points) On three consecutive days, number of ice-creams consumed
  were 1 0 2. If all three days had similar weather, which of HHH and
  CCC is more likely?
\item
  (10 points) Use Viterbi algorithm to find the maximum probability path
  and the maximum probability of this sequence.
\end{enumerate}

\begin{Shaded}
\begin{Highlighting}[]
\CommentTok{#Part A}
\NormalTok{HHH <-}\StringTok{ }\NormalTok{(}\FloatTok{0.5} \OperatorTok{*}\FloatTok{0.5}\NormalTok{)}\OperatorTok{*}\NormalTok{(}\FloatTok{0.7}\OperatorTok{*}\StringTok{ }\FloatTok{0.2}\NormalTok{)}\OperatorTok{*}\NormalTok{(}\FloatTok{0.7} \OperatorTok{*}\StringTok{ }\FloatTok{0.3}\NormalTok{)}
\NormalTok{HHH}
\end{Highlighting}
\end{Shaded}

\begin{verbatim}
## [1] 0.00735
\end{verbatim}

\begin{Shaded}
\begin{Highlighting}[]
\NormalTok{CCC <-}\StringTok{ }\NormalTok{(}\FloatTok{0.5} \OperatorTok{*}\FloatTok{0.3}\NormalTok{)}\OperatorTok{*}\NormalTok{(}\FloatTok{0.6}\OperatorTok{*}\StringTok{ }\FloatTok{0.6}\NormalTok{)}\OperatorTok{*}\NormalTok{(}\FloatTok{0.6} \OperatorTok{*}\FloatTok{0.1}\NormalTok{)}
\NormalTok{CCC}
\end{Highlighting}
\end{Shaded}

\begin{verbatim}
## [1] 0.00324
\end{verbatim}

\begin{Shaded}
\begin{Highlighting}[]
\CommentTok{##hot is more likely}

\CommentTok{#Part B}
\NormalTok{markov <-}\StringTok{ }\ControlFlowTok{function}\NormalTok{(x,P,n)\{ seq <-}\StringTok{ }\NormalTok{x}
\ControlFlowTok{for}\NormalTok{(k }\ControlFlowTok{in} \DecValTok{1}\OperatorTok{:}\NormalTok{(n}\DecValTok{-1}\NormalTok{))\{}
\NormalTok{seq[k}\OperatorTok{+}\DecValTok{1}\NormalTok{] <-}\StringTok{ }\KeywordTok{sample}\NormalTok{(x, }\DecValTok{1}\NormalTok{, }\DataTypeTok{replace=}\OtherTok{TRUE}\NormalTok{, P[seq[k],])\}}
\KeywordTok{return}\NormalTok{(seq)}
\NormalTok{\}}
\NormalTok{hmmdat <-}\StringTok{ }\ControlFlowTok{function}\NormalTok{(A,E,n)\{}
\NormalTok{observationset <-}\StringTok{ }\KeywordTok{c}\NormalTok{(}\DecValTok{1}\OperatorTok{:}\DecValTok{3}\NormalTok{) }\CommentTok{#3 diferent days}
\NormalTok{hiddenset <-}\StringTok{ }\KeywordTok{c}\NormalTok{(}\DecValTok{1}\NormalTok{,}\DecValTok{2}\NormalTok{) }\CommentTok{#hot denoted by 1, cold denoted by 2}
\NormalTok{x <-}\StringTok{ }\NormalTok{h <-}\StringTok{ }\KeywordTok{matrix}\NormalTok{(}\OtherTok{NA}\NormalTok{,}\DataTypeTok{nr=}\NormalTok{n,}\DataTypeTok{nc=}\DecValTok{1}\NormalTok{)}
\NormalTok{h[}\DecValTok{1}\NormalTok{]<-}\DecValTok{1}
\NormalTok{x[}\DecValTok{1}\NormalTok{]<-}\KeywordTok{sample}\NormalTok{(observationset,}\DecValTok{1}\NormalTok{,}\DataTypeTok{replace=}\OtherTok{TRUE}\NormalTok{,E[h[}\DecValTok{1}\NormalTok{],])}
\NormalTok{h <-}\StringTok{ }\KeywordTok{markov}\NormalTok{(hiddenset,A,n)}
\ControlFlowTok{for}\NormalTok{(k }\ControlFlowTok{in} \DecValTok{1}\OperatorTok{:}\NormalTok{(n}\DecValTok{-1}\NormalTok{))\{x[k}\OperatorTok{+}\DecValTok{1}\NormalTok{] <-}\StringTok{ }\KeywordTok{sample}\NormalTok{(observationset,}\DecValTok{1}\NormalTok{,}\DataTypeTok{replace=}\OtherTok{TRUE}\NormalTok{,E[h[k],])\}}
\NormalTok{out <-}\StringTok{ }\KeywordTok{matrix}\NormalTok{(}\KeywordTok{c}\NormalTok{(x,h),}\DataTypeTok{nrow=}\NormalTok{n,}\DataTypeTok{ncol=}\DecValTok{2}\NormalTok{,}\DataTypeTok{byrow=}\OtherTok{FALSE}\NormalTok{)}
\KeywordTok{return}\NormalTok{(out)}
\NormalTok{\}}

\NormalTok{E <-}\StringTok{ }\KeywordTok{matrix}\NormalTok{(}\KeywordTok{c}\NormalTok{(}\FloatTok{0.2}\NormalTok{,}\FloatTok{0.5}\NormalTok{,}\FloatTok{0.3}\NormalTok{,}\FloatTok{0.6}\NormalTok{,}\FloatTok{0.3}\NormalTok{,}\FloatTok{0.1}\NormalTok{),}\DecValTok{2}\NormalTok{,}\DecValTok{3}\NormalTok{,}\DataTypeTok{byrow=}\OtherTok{TRUE}\NormalTok{) }\CommentTok{#emission matrix}
\NormalTok{E}
\end{Highlighting}
\end{Shaded}

\begin{verbatim}
##      [,1] [,2] [,3]
## [1,]  0.2  0.5  0.3
## [2,]  0.6  0.3  0.1
\end{verbatim}

\begin{Shaded}
\begin{Highlighting}[]
\NormalTok{A <-}\StringTok{ }\KeywordTok{matrix}\NormalTok{(}\KeywordTok{c}\NormalTok{(}\FloatTok{0.7}\NormalTok{,}\FloatTok{0.3}\NormalTok{,}\FloatTok{0.6}\NormalTok{,}\FloatTok{0.4}\NormalTok{),}\DecValTok{2}\NormalTok{,}\DecValTok{2}\NormalTok{,}\DataTypeTok{byrow=}\OtherTok{TRUE}\NormalTok{) }\CommentTok{#transition matrix}
\NormalTok{A}
\end{Highlighting}
\end{Shaded}

\begin{verbatim}
##      [,1] [,2]
## [1,]  0.7  0.3
## [2,]  0.6  0.4
\end{verbatim}

\begin{Shaded}
\begin{Highlighting}[]
\NormalTok{dat <-}\StringTok{ }\KeywordTok{hmmdat}\NormalTok{(A,E,}\DecValTok{100}\NormalTok{)}
\KeywordTok{colnames}\NormalTok{(dat) <-}\StringTok{ }\KeywordTok{c}\NormalTok{(}\StringTok{"observation"}\NormalTok{,}\StringTok{"hidden_state"}\NormalTok{)}
\KeywordTok{rownames}\NormalTok{(dat) <-}\StringTok{ }\DecValTok{1}\OperatorTok{:}\DecValTok{100}
\KeywordTok{t}\NormalTok{(dat)}
\end{Highlighting}
\end{Shaded}

\begin{verbatim}
##              1 2 3 4 5 6 7 8 9 10 11 12 13 14 15 16 17 18 19 20 21 22 23 24 25
## observation  1 2 3 2 2 2 2 1 1  2  2  1  1  1  3  1  3  2  1  1  3  2  2  3  1
## hidden_state 1 1 1 1 1 1 2 2 1  1  2  1  2  1  1  1  1  2  1  1  2  1  1  1  1
##              26 27 28 29 30 31 32 33 34 35 36 37 38 39 40 41 42 43 44 45 46 47
## observation   3  3  2  1  2  2  2  1  2  2  3  3  3  2  1  1  3  3  1  2  2  2
## hidden_state  1  2  1  2  1  1  1  1  1  1  1  2  1  2  2  1  1  2  1  1  1  1
##              48 49 50 51 52 53 54 55 56 57 58 59 60 61 62 63 64 65 66 67 68 69
## observation   3  3  1  2  1  2  1  2  2  2  2  3  1  2  3  2  1  1  2  2  2  3
## hidden_state  1  2  1  2  1  2  2  1  1  1  1  1  1  1  1  2  2  1  1  1  1  1
##              70 71 72 73 74 75 76 77 78 79 80 81 82 83 84 85 86 87 88 89 90 91
## observation   2  1  1  2  3  1  1  3  3  1  1  3  1  1  1  2  1  3  1  2  1  1
## hidden_state  1  2  1  2  2  2  1  1  1  1  2  2  2  2  1  2  1  1  1  2  1  1
##              92 93 94 95 96 97 98 99 100
## observation   2  2  1  2  2  2  1  1   1
## hidden_state  1  2  1  1  1  2  2  2   1
\end{verbatim}

\begin{Shaded}
\begin{Highlighting}[]
\NormalTok{viterbi <-}\StringTok{ }\ControlFlowTok{function}\NormalTok{(A,E,x) \{}
\NormalTok{v <-}\StringTok{ }\KeywordTok{matrix}\NormalTok{(}\OtherTok{NA}\NormalTok{, }\DataTypeTok{nr=}\KeywordTok{length}\NormalTok{(x), }\DataTypeTok{nc=}\KeywordTok{dim}\NormalTok{(A)[}\DecValTok{1}\NormalTok{])}
\NormalTok{v[}\DecValTok{1}\NormalTok{,] <-}\StringTok{ }\DecValTok{0}\NormalTok{; v[}\DecValTok{1}\NormalTok{,}\DecValTok{1}\NormalTok{] <-}\StringTok{ }\DecValTok{1}
\ControlFlowTok{for}\NormalTok{(i }\ControlFlowTok{in} \DecValTok{2}\OperatorTok{:}\KeywordTok{length}\NormalTok{(x)) \{}
\ControlFlowTok{for}\NormalTok{ (l }\ControlFlowTok{in} \DecValTok{1}\OperatorTok{:}\KeywordTok{dim}\NormalTok{(A)[}\DecValTok{1}\NormalTok{]) \{v[i,l] <-}\StringTok{ }\NormalTok{E[l,x[i]] }\OperatorTok{*}\StringTok{ }\KeywordTok{max}\NormalTok{(v[(i}\DecValTok{-1}\NormalTok{),] }\OperatorTok{*}\StringTok{ }\NormalTok{A[l,])\}}
\NormalTok{\}}
\KeywordTok{return}\NormalTok{(v)}
\NormalTok{\}}
\NormalTok{vit <-}\StringTok{ }\KeywordTok{viterbi}\NormalTok{(A,E,dat[,}\DecValTok{1}\NormalTok{]) }\CommentTok{#using algorithm}
\NormalTok{vitrowmax <-}\StringTok{ }\KeywordTok{apply}\NormalTok{(vit, }\DecValTok{1}\NormalTok{, }\ControlFlowTok{function}\NormalTok{(x) }\KeywordTok{which.max}\NormalTok{(x)) }\CommentTok{#tracing back max prob p}
\NormalTok{datt <-}\StringTok{ }\KeywordTok{cbind}\NormalTok{(dat,vitrowmax)}
\KeywordTok{colnames}\NormalTok{(datt) <-}\StringTok{ }\KeywordTok{c}\NormalTok{(}\StringTok{"observation"}\NormalTok{,}\StringTok{"hidden_state"}\NormalTok{,}\StringTok{"predicted state"}\NormalTok{)}
\KeywordTok{t}\NormalTok{(datt)}
\end{Highlighting}
\end{Shaded}

\begin{verbatim}
##                 1 2 3 4 5 6 7 8 9 10 11 12 13 14 15 16 17 18 19 20 21 22 23 24
## observation     1 2 3 2 2 2 2 1 1  2  2  1  1  1  3  1  3  2  1  1  3  2  2  3
## hidden_state    1 1 1 1 1 1 2 2 1  1  2  1  2  1  1  1  1  2  1  1  2  1  1  1
## predicted state 1 1 1 1 1 1 1 2 2  1  1  2  2  2  1  2  1  1  2  2  1  1  1  1
##                 25 26 27 28 29 30 31 32 33 34 35 36 37 38 39 40 41 42 43 44 45
## observation      1  3  3  2  1  2  2  2  1  2  2  3  3  3  2  1  1  3  3  1  2
## hidden_state     1  1  2  1  2  1  1  1  1  1  1  1  2  1  2  2  1  1  2  1  1
## predicted state  2  1  1  1  2  1  1  1  2  1  1  1  1  1  1  2  2  1  1  2  1
##                 46 47 48 49 50 51 52 53 54 55 56 57 58 59 60 61 62 63 64 65 66
## observation      2  2  3  3  1  2  1  2  1  2  2  2  2  3  1  2  3  2  1  1  2
## hidden_state     1  1  1  2  1  2  1  2  2  1  1  1  1  1  1  1  1  2  2  1  1
## predicted state  1  1  1  1  2  1  2  1  2  1  1  1  1  1  2  1  1  1  2  2  1
##                 67 68 69 70 71 72 73 74 75 76 77 78 79 80 81 82 83 84 85 86 87
## observation      2  2  3  2  1  1  2  3  1  1  3  3  1  1  3  1  1  1  2  1  3
## hidden_state     1  1  1  1  2  1  2  2  2  1  1  1  1  2  2  2  2  1  2  1  1
## predicted state  1  1  1  1  2  2  1  1  2  2  1  1  2  2  1  2  2  2  1  2  1
##                 88 89 90 91 92 93 94 95 96 97 98 99 100
## observation      1  2  1  1  2  2  1  2  2  2  1  1   1
## hidden_state     1  2  1  1  1  2  1  1  1  2  2  2   1
## predicted state  2  1  2  2  1  1  2  1  1  1  2  2   2
\end{verbatim}

\begin{Shaded}
\begin{Highlighting}[]
\NormalTok{hiddenstate <-}\StringTok{ }\NormalTok{dat[,}\DecValTok{2}\NormalTok{]}
\NormalTok{tab<-}\KeywordTok{table}\NormalTok{(hiddenstate, vitrowmax)}
\NormalTok{accuracy<-}\KeywordTok{sum}\NormalTok{(}\KeywordTok{diag}\NormalTok{(tab))}\OperatorTok{/}\KeywordTok{sum}\NormalTok{(tab)}
\NormalTok{misClasificerror<-}\DecValTok{1}\OperatorTok{-}\NormalTok{accuracy}
\NormalTok{misClasificerror}
\end{Highlighting}
\end{Shaded}

\begin{verbatim}
## [1] 0.43
\end{verbatim}

4.This question has two related parts. (a) (10 points) Consider the 4
nucleotides of a DNA sequence `a', `c', `g', `t'. For these 4 states,
keeping in mind properties of a transition matrix, define any 4 by 4
transition matrix of your choice. Now, using this transition matrix of
your choice, P, and an initial probability vector 𝜋 = (0.4,0.1,0.1,0.4),
generate a Markov chain of length 1000. (b) (10 points) For the chain
generated in (a), use a chi-square goodness of fit test to check if
purines (`a', `g') and pyrimidines (`c', `t') have equal probability.

\begin{Shaded}
\begin{Highlighting}[]
\KeywordTok{library}\NormalTok{(ade4)}
\end{Highlighting}
\end{Shaded}

\begin{verbatim}
## Warning: package 'ade4' was built under R version 4.0.4
\end{verbatim}

\begin{Shaded}
\begin{Highlighting}[]
\KeywordTok{library}\NormalTok{(seqinr)}
\end{Highlighting}
\end{Shaded}

\begin{verbatim}
## Warning: package 'seqinr' was built under R version 4.0.4
\end{verbatim}

\begin{verbatim}
## 
## Attaching package: 'seqinr'
\end{verbatim}

\begin{verbatim}
## The following object is masked from 'package:caret':
## 
##     dotPlot
\end{verbatim}

\begin{Shaded}
\begin{Highlighting}[]
\NormalTok{x<-}\KeywordTok{read.fasta}\NormalTok{(}\StringTok{"C:/Users/willi/OneDrive/Documents/E.Coli/AE005174v2.fas"}\NormalTok{)}
\NormalTok{ecoli<-}\KeywordTok{c}\NormalTok{(x[[}\DecValTok{1}\NormalTok{]],x[[}\DecValTok{2}\NormalTok{]])}
\NormalTok{P<-}\KeywordTok{prop.table}\NormalTok{(}\KeywordTok{table}\NormalTok{(ecoli[}\DecValTok{4000}\OperatorTok{:}\DecValTok{4999}\NormalTok{],ecoli[}\DecValTok{4001}\OperatorTok{:}\DecValTok{5000}\NormalTok{]),}\DecValTok{1}\NormalTok{)}
\NormalTok{P }\CommentTok{#Transition matrix}
\end{Highlighting}
\end{Shaded}

\begin{verbatim}
##    
##             a         c         g         t
##   a 0.3333333 0.2294372 0.2164502 0.2207792
##   c 0.2240000 0.2040000 0.3560000 0.2160000
##   g 0.2379310 0.3379310 0.2172414 0.2068966
##   t 0.1266376 0.2096070 0.3799127 0.2838428
\end{verbatim}

\begin{Shaded}
\begin{Highlighting}[]
\NormalTok{pi<-}\KeywordTok{c}\NormalTok{(}\FloatTok{0.4}\NormalTok{,}\FloatTok{0.1}\NormalTok{,}\FloatTok{0.1}\NormalTok{,}\FloatTok{0.4}\NormalTok{)}
\NormalTok{nucleotides <-}\StringTok{ }\KeywordTok{c}\NormalTok{(}\StringTok{"a"}\NormalTok{,}\StringTok{"c"}\NormalTok{,}\StringTok{"t"}\NormalTok{,}\StringTok{"g"}\NormalTok{)}
\NormalTok{length<-}\DecValTok{1000}
\NormalTok{chain<-}\KeywordTok{rep}\NormalTok{(}\OtherTok{NA}\NormalTok{,length)}
\NormalTok{chain[}\DecValTok{1}\NormalTok{]<-}\KeywordTok{sample}\NormalTok{(nucleotides,}\DecValTok{1}\NormalTok{,}\DataTypeTok{p=}\NormalTok{pi)}
\NormalTok{chain[}\DecValTok{1}\NormalTok{]}
\end{Highlighting}
\end{Shaded}

\begin{verbatim}
## [1] "t"
\end{verbatim}

\begin{Shaded}
\begin{Highlighting}[]
\ControlFlowTok{for}\NormalTok{ (i }\ControlFlowTok{in} \DecValTok{1}\OperatorTok{:}\DecValTok{999}\NormalTok{)\{}
\NormalTok{ chain[i}\OperatorTok{+}\DecValTok{1}\NormalTok{]<-}\KeywordTok{sample}\NormalTok{(nucleotides,}\DecValTok{1}\NormalTok{,}\DataTypeTok{p=}\NormalTok{P[chain[i],])}
\NormalTok{ \}}
\KeywordTok{head}\NormalTok{(chain)}
\end{Highlighting}
\end{Shaded}

\begin{verbatim}
## [1] "t" "t" "t" "t" "t" "c"
\end{verbatim}

\begin{Shaded}
\begin{Highlighting}[]
\CommentTok{#b)}
\NormalTok{x<-}\KeywordTok{table}\NormalTok{(chain[}\DecValTok{1}\OperatorTok{:}\DecValTok{1000}\NormalTok{])[}\KeywordTok{c}\NormalTok{(}\StringTok{"a"}\NormalTok{,}\StringTok{"c"}\NormalTok{,}\StringTok{"t"}\NormalTok{,}\StringTok{"g"}\NormalTok{)]}
\NormalTok{x<-}\KeywordTok{c}\NormalTok{(}\DecValTok{2}\OperatorTok{*}\NormalTok{(x[}\DecValTok{4}\NormalTok{]}\OperatorTok{+}\NormalTok{x[}\DecValTok{2}\NormalTok{]), }\DecValTok{2}\OperatorTok{*}\NormalTok{(x[}\DecValTok{1}\NormalTok{]}\OperatorTok{+}\NormalTok{x[}\DecValTok{3}\NormalTok{]))}\CommentTok{#freq of G, #freq of others}
\NormalTok{p<-}\KeywordTok{c}\NormalTok{(}\FloatTok{0.5}\NormalTok{,}\FloatTok{0.5}\NormalTok{)}
\KeywordTok{chisq.test}\NormalTok{(x,}\DataTypeTok{p=}\NormalTok{p)}
\end{Highlighting}
\end{Shaded}

\begin{verbatim}
## 
##  Chi-squared test for given probabilities
## 
## data:  x
## X-squared = 2.048, df = 1, p-value = 0.1524
\end{verbatim}

\begin{enumerate}
\def\labelenumi{\arabic{enumi}.}
\setcounter{enumi}{5}
\item
  \begin{enumerate}
  \def\labelenumii{(\alph{enumii})}
  \tightlist
  \item
    (5 points) For your data set `final', perform principal component
    analysis on the 11 predictor variables only. (leave out quality of
    wine for this part)
  \end{enumerate}
\end{enumerate}

\begin{enumerate}
\def\labelenumi{(\alph{enumi})}
\setcounter{enumi}{1}
\tightlist
\item
  (2+2+2 points) To retain at least 90\% variation in the data, how many
  principal components (PCs) should be retained? How much variation are
  you retaining exactly here? Justify why you see so many or so few PCs
  to retain 90\% variation.
\item
  (3 points) Express the first PC as a linear combination of the 11
  original variables. Comment on these coefficients.
\item
  (2 points) Should there be any correlation between PC1 and PC2?
  Justify.
\item
  (4 points) Plot PC1 vs PC2 and overlap quality of wine (`high' and
  `low') with different colors. Comment.
\end{enumerate}

\begin{Shaded}
\begin{Highlighting}[]
\NormalTok{pca<-}\StringTok{ }\KeywordTok{prcomp}\NormalTok{(final, }\DataTypeTok{center =} \OtherTok{TRUE}\NormalTok{,}\DataTypeTok{scale. =} \OtherTok{TRUE}\NormalTok{)}
\KeywordTok{summary}\NormalTok{(pca)}
\end{Highlighting}
\end{Shaded}

\begin{verbatim}
## Importance of components:
##                           PC1    PC2    PC3     PC4     PC5     PC6    PC7
## Standard deviation     1.8435 1.2547 1.1355 1.02792 0.99281 0.95698 0.8846
## Proportion of Variance 0.2832 0.1312 0.1075 0.08805 0.08214 0.07632 0.0652
## Cumulative Proportion  0.2832 0.4144 0.5219 0.60990 0.69204 0.76835 0.8336
##                            PC8     PC9    PC10    PC11    PC12
## Standard deviation     0.86507 0.75453 0.61533 0.53505 0.12136
## Proportion of Variance 0.06236 0.04744 0.03155 0.02386 0.00123
## Cumulative Proportion  0.89592 0.94336 0.97492 0.99877 1.00000
\end{verbatim}

\begin{Shaded}
\begin{Highlighting}[]
\CommentTok{#standardization is required for PCA}
\CommentTok{#So let us use the correlation matrix instead of the variance-covariance matrix}
\NormalTok{R<-}\KeywordTok{cor}\NormalTok{(}\KeywordTok{t}\NormalTok{(final))}
\KeywordTok{dim}\NormalTok{(R)}
\end{Highlighting}
\end{Shaded}

\begin{verbatim}
## [1] 1500 1500
\end{verbatim}

\begin{Shaded}
\begin{Highlighting}[]
\NormalTok{E<-}\KeywordTok{eigen}\NormalTok{(R) }

\CommentTok{####}
\NormalTok{values<-E}\OperatorTok{$}\NormalTok{values}
\KeywordTok{sum}\NormalTok{(values[}\DecValTok{1}\NormalTok{])}\OperatorTok{/}\KeywordTok{sum}\NormalTok{(values)}
\end{Highlighting}
\end{Shaded}

\begin{verbatim}
## [1] 0.9878589
\end{verbatim}

\begin{Shaded}
\begin{Highlighting}[]
\CommentTok{#BIPLOT}
\KeywordTok{library}\NormalTok{(ggfortify)}
\end{Highlighting}
\end{Shaded}

\begin{verbatim}
## Warning: package 'ggfortify' was built under R version 4.0.5
\end{verbatim}

\begin{Shaded}
\begin{Highlighting}[]
\KeywordTok{autoplot}\NormalTok{(pca, }\DataTypeTok{data =}\NormalTok{ final, }\DataTypeTok{loadings =} \OtherTok{TRUE}\NormalTok{, }\DataTypeTok{loadings.label=}\OtherTok{TRUE}\NormalTok{)}
\end{Highlighting}
\end{Shaded}

\begin{verbatim}
## Warning: `select_()` was deprecated in dplyr 0.7.0.
## Please use `select()` instead.
\end{verbatim}

\includegraphics{Final337_files/figure-latex/unnamed-chunk-5-1.pdf}

\begin{Shaded}
\begin{Highlighting}[]
\CommentTok{#is there clustering with respect to Species?}
\KeywordTok{autoplot}\NormalTok{(pca, }\DataTypeTok{data =}\NormalTok{ final, }\DataTypeTok{loadings =} \OtherTok{TRUE}\NormalTok{, }\DataTypeTok{loadings.label=}\OtherTok{TRUE}\NormalTok{, }\DataTypeTok{colour=}\StringTok{'quality'}\NormalTok{)}
\end{Highlighting}
\end{Shaded}

\includegraphics{Final337_files/figure-latex/unnamed-chunk-5-2.pdf}

\begin{Shaded}
\begin{Highlighting}[]
\KeywordTok{biplot}\NormalTok{(pca)}
\end{Highlighting}
\end{Shaded}

\includegraphics{Final337_files/figure-latex/unnamed-chunk-5-3.pdf}

\begin{Shaded}
\begin{Highlighting}[]
\KeywordTok{autoplot}\NormalTok{(pca, }\DataTypeTok{data =}\NormalTok{ final, }\DataTypeTok{loadings =} \OtherTok{TRUE}\NormalTok{, }\DataTypeTok{loadings.label=}\OtherTok{TRUE}\NormalTok{, }\DataTypeTok{colour=}\StringTok{'quality'}\NormalTok{, }\DataTypeTok{frame=}\OtherTok{TRUE}\NormalTok{, }\DataTypeTok{frame.type=}\StringTok{'norm'}\NormalTok{)}
\end{Highlighting}
\end{Shaded}

\includegraphics{Final337_files/figure-latex/unnamed-chunk-5-4.pdf}

\end{document}
