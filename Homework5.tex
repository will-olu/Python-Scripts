% Options for packages loaded elsewhere
\PassOptionsToPackage{unicode}{hyperref}
\PassOptionsToPackage{hyphens}{url}
%
\documentclass[
]{article}
\usepackage{lmodern}
\usepackage{amssymb,amsmath}
\usepackage{ifxetex,ifluatex}
\ifnum 0\ifxetex 1\fi\ifluatex 1\fi=0 % if pdftex
  \usepackage[T1]{fontenc}
  \usepackage[utf8]{inputenc}
  \usepackage{textcomp} % provide euro and other symbols
\else % if luatex or xetex
  \usepackage{unicode-math}
  \defaultfontfeatures{Scale=MatchLowercase}
  \defaultfontfeatures[\rmfamily]{Ligatures=TeX,Scale=1}
\fi
% Use upquote if available, for straight quotes in verbatim environments
\IfFileExists{upquote.sty}{\usepackage{upquote}}{}
\IfFileExists{microtype.sty}{% use microtype if available
  \usepackage[]{microtype}
  \UseMicrotypeSet[protrusion]{basicmath} % disable protrusion for tt fonts
}{}
\makeatletter
\@ifundefined{KOMAClassName}{% if non-KOMA class
  \IfFileExists{parskip.sty}{%
    \usepackage{parskip}
  }{% else
    \setlength{\parindent}{0pt}
    \setlength{\parskip}{6pt plus 2pt minus 1pt}}
}{% if KOMA class
  \KOMAoptions{parskip=half}}
\makeatother
\usepackage{xcolor}
\IfFileExists{xurl.sty}{\usepackage{xurl}}{} % add URL line breaks if available
\IfFileExists{bookmark.sty}{\usepackage{bookmark}}{\usepackage{hyperref}}
\hypersetup{
  pdftitle={Homework \#5},
  pdfauthor={Ade Olu-Ajeigbe},
  hidelinks,
  pdfcreator={LaTeX via pandoc}}
\urlstyle{same} % disable monospaced font for URLs
\usepackage[margin=1in]{geometry}
\usepackage{color}
\usepackage{fancyvrb}
\newcommand{\VerbBar}{|}
\newcommand{\VERB}{\Verb[commandchars=\\\{\}]}
\DefineVerbatimEnvironment{Highlighting}{Verbatim}{commandchars=\\\{\}}
% Add ',fontsize=\small' for more characters per line
\usepackage{framed}
\definecolor{shadecolor}{RGB}{248,248,248}
\newenvironment{Shaded}{\begin{snugshade}}{\end{snugshade}}
\newcommand{\AlertTok}[1]{\textcolor[rgb]{0.94,0.16,0.16}{#1}}
\newcommand{\AnnotationTok}[1]{\textcolor[rgb]{0.56,0.35,0.01}{\textbf{\textit{#1}}}}
\newcommand{\AttributeTok}[1]{\textcolor[rgb]{0.77,0.63,0.00}{#1}}
\newcommand{\BaseNTok}[1]{\textcolor[rgb]{0.00,0.00,0.81}{#1}}
\newcommand{\BuiltInTok}[1]{#1}
\newcommand{\CharTok}[1]{\textcolor[rgb]{0.31,0.60,0.02}{#1}}
\newcommand{\CommentTok}[1]{\textcolor[rgb]{0.56,0.35,0.01}{\textit{#1}}}
\newcommand{\CommentVarTok}[1]{\textcolor[rgb]{0.56,0.35,0.01}{\textbf{\textit{#1}}}}
\newcommand{\ConstantTok}[1]{\textcolor[rgb]{0.00,0.00,0.00}{#1}}
\newcommand{\ControlFlowTok}[1]{\textcolor[rgb]{0.13,0.29,0.53}{\textbf{#1}}}
\newcommand{\DataTypeTok}[1]{\textcolor[rgb]{0.13,0.29,0.53}{#1}}
\newcommand{\DecValTok}[1]{\textcolor[rgb]{0.00,0.00,0.81}{#1}}
\newcommand{\DocumentationTok}[1]{\textcolor[rgb]{0.56,0.35,0.01}{\textbf{\textit{#1}}}}
\newcommand{\ErrorTok}[1]{\textcolor[rgb]{0.64,0.00,0.00}{\textbf{#1}}}
\newcommand{\ExtensionTok}[1]{#1}
\newcommand{\FloatTok}[1]{\textcolor[rgb]{0.00,0.00,0.81}{#1}}
\newcommand{\FunctionTok}[1]{\textcolor[rgb]{0.00,0.00,0.00}{#1}}
\newcommand{\ImportTok}[1]{#1}
\newcommand{\InformationTok}[1]{\textcolor[rgb]{0.56,0.35,0.01}{\textbf{\textit{#1}}}}
\newcommand{\KeywordTok}[1]{\textcolor[rgb]{0.13,0.29,0.53}{\textbf{#1}}}
\newcommand{\NormalTok}[1]{#1}
\newcommand{\OperatorTok}[1]{\textcolor[rgb]{0.81,0.36,0.00}{\textbf{#1}}}
\newcommand{\OtherTok}[1]{\textcolor[rgb]{0.56,0.35,0.01}{#1}}
\newcommand{\PreprocessorTok}[1]{\textcolor[rgb]{0.56,0.35,0.01}{\textit{#1}}}
\newcommand{\RegionMarkerTok}[1]{#1}
\newcommand{\SpecialCharTok}[1]{\textcolor[rgb]{0.00,0.00,0.00}{#1}}
\newcommand{\SpecialStringTok}[1]{\textcolor[rgb]{0.31,0.60,0.02}{#1}}
\newcommand{\StringTok}[1]{\textcolor[rgb]{0.31,0.60,0.02}{#1}}
\newcommand{\VariableTok}[1]{\textcolor[rgb]{0.00,0.00,0.00}{#1}}
\newcommand{\VerbatimStringTok}[1]{\textcolor[rgb]{0.31,0.60,0.02}{#1}}
\newcommand{\WarningTok}[1]{\textcolor[rgb]{0.56,0.35,0.01}{\textbf{\textit{#1}}}}
\usepackage{graphicx,grffile}
\makeatletter
\def\maxwidth{\ifdim\Gin@nat@width>\linewidth\linewidth\else\Gin@nat@width\fi}
\def\maxheight{\ifdim\Gin@nat@height>\textheight\textheight\else\Gin@nat@height\fi}
\makeatother
% Scale images if necessary, so that they will not overflow the page
% margins by default, and it is still possible to overwrite the defaults
% using explicit options in \includegraphics[width, height, ...]{}
\setkeys{Gin}{width=\maxwidth,height=\maxheight,keepaspectratio}
% Set default figure placement to htbp
\makeatletter
\def\fps@figure{htbp}
\makeatother
\setlength{\emergencystretch}{3em} % prevent overfull lines
\providecommand{\tightlist}{%
  \setlength{\itemsep}{0pt}\setlength{\parskip}{0pt}}
\setcounter{secnumdepth}{-\maxdimen} % remove section numbering

\title{Homework \#5}
\author{Ade Olu-Ajeigbe}
\date{4/30/2021}

\begin{document}
\maketitle

\begin{enumerate}
\def\labelenumi{\arabic{enumi}.}
\tightlist
\item
  We are using a Hidden Markov model for segmenting genomic DNA
  sequences into exons
\end{enumerate}

\begin{enumerate}
\def\labelenumi{(\Alph{enumi})}
\setcounter{enumi}{4}
\tightlist
\item
  and introns (I). Consider the following HMM and a recognition site
  CGTA
\end{enumerate}

\begin{Shaded}
\begin{Highlighting}[]
\CommentTok{#We are using a Hidden Markov model for segmenting genomic DNA sequences into exons}
\CommentTok{#(E) and introns (I). Consider the following HMM and a recognition site CGTA}

\KeywordTok{library}\NormalTok{(tinytex)}
\end{Highlighting}
\end{Shaded}

\begin{verbatim}
## Warning: package 'tinytex' was built under R version 4.0.5
\end{verbatim}

\begin{Shaded}
\begin{Highlighting}[]
\CommentTok{#probabilities}
\CommentTok{#P(A|E) = 0.2 #Given that E has already occured the probability of A is 0.2}
\CommentTok{#P(C|E) = 0.4 #Given that E has already occured the probability of C is 0.4}
\CommentTok{#P(G|E) = 0.4 #Given that E has already occured the probability of G is 0.4}
\CommentTok{#P(t|E) = 0.2 #Given that E has occured the probability of T is 0.2}
\CommentTok{#P(A|I) = 0.4 #Given that I has occured the probability of A is 0.4}
\CommentTok{#P(C|I) = 0.2 #Given that I has occured the probability  of C is 0.2}
\CommentTok{#P(G|I) = 0.2 #Given that I has occured the probability of G is 0.2}
\CommentTok{#P(t|I) = 0.4 #Given that I has occured the probability of T is 0.4}
\CommentTok{#P(I given E) = 0.3 transtion of E to I}
\CommentTok{#P(E|I) = 0.4 tranisiton of I to E}
\CommentTok{#P(E|E) = 0.6}
\CommentTok{#P(I|I) = 0.7}
\CommentTok{#transition probilities are 0.3, 0.4, 0.6, 0.7}
\CommentTok{#for E the emission probabilities at a, c, g, t are 0.2, 0.4 , 0.4, 0.2}
\CommentTok{#for I the emission probabilityies at a, c, g, t are 0.4, 0.2, 0.2, 0.4}
\CommentTok{#the sequence is c to g to a to t}

\CommentTok{#number of states is equal to 2 so k is equal to 2 }
\CommentTok{#so 1/k is 1/2}
\NormalTok{a0I =}\StringTok{ }\DecValTok{1}\OperatorTok{/}\DecValTok{2}
\NormalTok{a0E =}\StringTok{ }\DecValTok{1}\OperatorTok{/}\DecValTok{2}

\CommentTok{#x<-c(((1/2) * 0.4) *(0.6* 0.4)*(0.6 *0.2) *(0.6*0.2)) this is indepth test of the specific recognition site c g t a for an all exon model condensed below in the EEEE function}
\CommentTok{#x}
\NormalTok{EEEE<-}\KeywordTok{c}\NormalTok{((}\DecValTok{1}\OperatorTok{/}\DecValTok{2}\NormalTok{)}\OperatorTok{*}\NormalTok{(}\FloatTok{0.6}\OperatorTok{^}\DecValTok{3}\NormalTok{)}\OperatorTok{*}\NormalTok{(}\FloatTok{0.1}\OperatorTok{^}\DecValTok{2}\NormalTok{)}\OperatorTok{*}\NormalTok{(}\FloatTok{0.4}\OperatorTok{^}\DecValTok{2}\NormalTok{))}
\NormalTok{EEEE}
\end{Highlighting}
\end{Shaded}

\begin{verbatim}
## [1] 0.0001728
\end{verbatim}

\begin{Shaded}
\begin{Highlighting}[]
\NormalTok{IIII <-}\KeywordTok{c}\NormalTok{((}\DecValTok{1}\OperatorTok{/}\DecValTok{2}\NormalTok{)}\OperatorTok{*}\NormalTok{(}\FloatTok{0.7}\OperatorTok{*}\DecValTok{3}\NormalTok{)}\OperatorTok{*}\NormalTok{(}\FloatTok{0.1}\OperatorTok{^}\DecValTok{2}\NormalTok{)}\OperatorTok{*}\NormalTok{(}\FloatTok{0.4}\OperatorTok{^}\DecValTok{2}\NormalTok{))}
\NormalTok{IIII}
\end{Highlighting}
\end{Shaded}

\begin{verbatim}
## [1] 0.00168
\end{verbatim}

\begin{Shaded}
\begin{Highlighting}[]
\CommentTok{##IIII is more likely}
\end{Highlighting}
\end{Shaded}

\#3. (a) Use R to generate a sequence of length 200 from this Hidden
Markov model. \#(b) Now use Viterbi algorithm to decode the state (H/L)
for the 200 outcomes \#{[}For parts (a) and (b) refer to R code
``Viterbi Algorithm \_Dishonest Casino Example''. \#Here the simulation
and Viterbi has been done for the Dishonest Casino die problem. You
\#need to modify it accordingly{]}

\begin{Shaded}
\begin{Highlighting}[]
\CommentTok{#maybe specify that the length of the chain is 200}
\NormalTok{n<-(}\KeywordTok{c}\NormalTok{(}\DecValTok{200}\NormalTok{))}


\CommentTok{#This is a matrix  visualizing the }
\NormalTok{transition<-}\StringTok{ }\KeywordTok{matrix}\NormalTok{(}\KeywordTok{c}\NormalTok{(}\FloatTok{0.6}\NormalTok{, }\FloatTok{0.3}\NormalTok{, }\FloatTok{0.4}\NormalTok{, }\FloatTok{0.7}\NormalTok{),}\DataTypeTok{nrow =} \DecValTok{2}\NormalTok{, }\DataTypeTok{ncol=} \DecValTok{2}\NormalTok{)}
\KeywordTok{rownames}\NormalTok{(transition) <-}\StringTok{ }\KeywordTok{c}\NormalTok{(}\StringTok{"E"}\NormalTok{, }\StringTok{"I"}\NormalTok{)}
\KeywordTok{colnames}\NormalTok{(transition) <-}\StringTok{ }\KeywordTok{c}\NormalTok{(}\StringTok{"E"}\NormalTok{, }\StringTok{"I"}\NormalTok{)}
\NormalTok{A <-}\StringTok{ }\NormalTok{transition}
\NormalTok{A}
\end{Highlighting}
\end{Shaded}

\begin{verbatim}
##     E   I
## E 0.6 0.4
## I 0.3 0.7
\end{verbatim}

\begin{Shaded}
\begin{Highlighting}[]
\NormalTok{Enucs<-}\StringTok{ }\KeywordTok{c}\NormalTok{(}\DecValTok{1}\NormalTok{,}\DecValTok{1}\NormalTok{,}\DecValTok{4}\NormalTok{,}\DecValTok{4}\NormalTok{)}
\NormalTok{Inucs<-}\StringTok{ }\KeywordTok{c}\NormalTok{(}\DecValTok{4}\NormalTok{,}\DecValTok{4}\NormalTok{,}\DecValTok{1}\NormalTok{,}\DecValTok{1}\NormalTok{)}
\NormalTok{E<-}\KeywordTok{matrix}\NormalTok{(}\KeywordTok{c}\NormalTok{(}\KeywordTok{rep}\NormalTok{(Enucs}\OperatorTok{/}\DecValTok{10}\NormalTok{),}\KeywordTok{rep}\NormalTok{(Inucs}\OperatorTok{/}\DecValTok{10}\NormalTok{)),}\DataTypeTok{nrow=}\DecValTok{2}\NormalTok{,}\DataTypeTok{ncol =} \DecValTok{4}\NormalTok{, }\DataTypeTok{byrow =} \OtherTok{TRUE}\NormalTok{)}
\NormalTok{E}
\end{Highlighting}
\end{Shaded}

\begin{verbatim}
##      [,1] [,2] [,3] [,4]
## [1,]  0.1  0.1  0.4  0.4
## [2,]  0.4  0.4  0.1  0.1
\end{verbatim}

\begin{Shaded}
\begin{Highlighting}[]
\NormalTok{markov <-}\StringTok{ }\ControlFlowTok{function}\NormalTok{(x,P,n)\{ seq <-}\StringTok{ }\NormalTok{x}
\ControlFlowTok{for}\NormalTok{(k }\ControlFlowTok{in} \DecValTok{1}\OperatorTok{:}\NormalTok{(n}\DecValTok{-1}\NormalTok{))\{}
\NormalTok{ seq[k}\OperatorTok{+}\DecValTok{1}\NormalTok{] <-}\StringTok{ }\KeywordTok{sample}\NormalTok{(x, }\DecValTok{1}\NormalTok{, }\DataTypeTok{replace=}\OtherTok{TRUE}\NormalTok{, P[seq[k],])\}}
\KeywordTok{return}\NormalTok{(seq)}
\NormalTok{\}}

\NormalTok{nucleotides <-(}\KeywordTok{c}\NormalTok{(}\StringTok{"a"}\NormalTok{,}\StringTok{"c"}\NormalTok{,}\StringTok{"g"}\NormalTok{,}\StringTok{"t"}\NormalTok{))}

\CommentTok{#here you need to manipulate the data to have the observation set and hiddenset}
\NormalTok{hmmdat <-}\StringTok{ }\ControlFlowTok{function}\NormalTok{(A,E,n)\{}
\NormalTok{ observationset <-}\StringTok{ }\NormalTok{nucleotides[}\DecValTok{1}\OperatorTok{:}\DecValTok{4}\NormalTok{] }\CommentTok{# a, c , g, t}
\NormalTok{ hiddenset <-}\StringTok{ }\KeywordTok{c}\NormalTok{(}\DecValTok{1}\NormalTok{,}\DecValTok{2}\NormalTok{) }\CommentTok{#intron is 1 and exon is 2 since E is defined as our emission}
\NormalTok{ x <-}\StringTok{ }\NormalTok{h <-}\StringTok{ }\KeywordTok{matrix}\NormalTok{(}\OtherTok{NA}\NormalTok{,}\DataTypeTok{nr=}\NormalTok{n,}\DataTypeTok{nc=}\DecValTok{1}\NormalTok{)}
\NormalTok{ h[}\DecValTok{1}\NormalTok{]<-}\DecValTok{1}
\NormalTok{ x[}\DecValTok{1}\NormalTok{]<-}\KeywordTok{sample}\NormalTok{(observationset,}\DecValTok{1}\NormalTok{,}\DataTypeTok{replace=}\OtherTok{TRUE}\NormalTok{,E[h[}\DecValTok{1}\NormalTok{],])}
\NormalTok{ h <-}\StringTok{ }\KeywordTok{markov}\NormalTok{(hiddenset,A,n)}
 \ControlFlowTok{for}\NormalTok{(k }\ControlFlowTok{in} \DecValTok{1}\OperatorTok{:}\NormalTok{(n}\DecValTok{-1}\NormalTok{))\{x[k}\OperatorTok{+}\DecValTok{1}\NormalTok{] <-}\StringTok{ }\KeywordTok{sample}\NormalTok{(observationset,}\DecValTok{1}\NormalTok{,}\DataTypeTok{replace=}\OtherTok{TRUE}\NormalTok{,E[h[k],])\}}
\NormalTok{ out <-}\StringTok{ }\KeywordTok{matrix}\NormalTok{(}\KeywordTok{c}\NormalTok{(x,h),}\DataTypeTok{nrow=}\NormalTok{n,}\DataTypeTok{ncol=}\DecValTok{2}\NormalTok{,}\DataTypeTok{byrow=}\OtherTok{FALSE}\NormalTok{)}
 \KeywordTok{return}\NormalTok{(out)}
\NormalTok{\}}


\CommentTok{#length of Hidden Markov model is 200}
\NormalTok{dat <-}\StringTok{ }\KeywordTok{hmmdat}\NormalTok{(A,E, n)}
\KeywordTok{colnames}\NormalTok{(dat) <-}\StringTok{ }\KeywordTok{c}\NormalTok{(}\StringTok{"observation"}\NormalTok{,}\StringTok{"hidden_state"}\NormalTok{)}
\KeywordTok{rownames}\NormalTok{(dat) <-}\StringTok{ }\DecValTok{1}\OperatorTok{:}\NormalTok{n}
\KeywordTok{t}\NormalTok{(dat)}
\end{Highlighting}
\end{Shaded}

\begin{verbatim}
##              1   2   3   4   5   6   7   8   9   10  11  12  13  14  15  16 
## observation  "g" "g" "c" "g" "t" "t" "g" "t" "g" "a" "c" "a" "a" "g" "g" "a"
## hidden_state "1" "2" "2" "2" "1" "1" "1" "1" "2" "2" "2" "2" "1" "1" "2" "1"
##              17  18  19  20  21  22  23  24  25  26  27  28  29  30  31  32 
## observation  "t" "t" "a" "g" "g" "a" "c" "c" "t" "t" "g" "t" "t" "a" "c" "c"
## hidden_state "1" "2" "1" "1" "2" "2" "2" "1" "1" "1" "2" "1" "2" "2" "2" "2"
##              33  34  35  36  37  38  39  40  41  42  43  44  45  46  47  48 
## observation  "a" "c" "t" "c" "t" "t" "g" "t" "t" "t" "a" "g" "a" "a" "g" "t"
## hidden_state "2" "2" "2" "1" "1" "1" "2" "2" "1" "2" "2" "2" "2" "2" "1" "1"
##              49  50  51  52  53  54  55  56  57  58  59  60  61  62  63  64 
## observation  "t" "g" "g" "t" "t" "t" "c" "c" "a" "a" "a" "c" "c" "c" "t" "g"
## hidden_state "1" "2" "1" "1" "1" "2" "2" "2" "2" "2" "2" "2" "2" "1" "1" "1"
##              65  66  67  68  69  70  71  72  73  74  75  76  77  78  79  80 
## observation  "a" "a" "t" "t" "g" "g" "t" "a" "a" "a" "c" "t" "t" "c" "a" "c"
## hidden_state "2" "1" "1" "1" "1" "1" "2" "2" "2" "2" "1" "1" "2" "2" "1" "2"
##              81  82  83  84  85  86  87  88  89  90  91  92  93  94  95  96 
## observation  "c" "g" "a" "a" "a" "t" "a" "a" "c" "c" "a" "g" "t" "t" "c" "t"
## hidden_state "1" "2" "2" "2" "1" "2" "2" "2" "1" "2" "2" "2" "2" "2" "2" "2"
##              97  98  99  100 101 102 103 104 105 106 107 108 109 110 111 112
## observation  "a" "c" "c" "c" "t" "c" "t" "t" "a" "a" "t" "a" "c" "a" "t" "c"
## hidden_state "1" "2" "2" "1" "2" "1" "1" "2" "2" "2" "2" "2" "2" "2" "2" "2"
##              113 114 115 116 117 118 119 120 121 122 123 124 125 126 127 128
## observation  "t" "g" "c" "c" "g" "c" "t" "c" "a" "t" "c" "a" "t" "g" "a" "g"
## hidden_state "1" "2" "2" "1" "2" "1" "2" "2" "1" "2" "2" "1" "1" "2" "2" "1"
##              129 130 131 132 133 134 135 136 137 138 139 140 141 142 143 144
## observation  "t" "t" "g" "c" "a" "a" "g" "g" "g" "t" "t" "g" "g" "a" "c" "a"
## hidden_state "1" "1" "2" "2" "2" "1" "1" "2" "2" "1" "1" "1" "2" "2" "2" "2"
##              145 146 147 148 149 150 151 152 153 154 155 156 157 158 159 160
## observation  "c" "a" "a" "g" "g" "g" "t" "g" "t" "t" "c" "c" "c" "t" "c" "t"
## hidden_state "2" "2" "2" "2" "2" "1" "1" "1" "1" "1" "1" "2" "1" "2" "2" "1"
##              161 162 163 164 165 166 167 168 169 170 171 172 173 174 175 176
## observation  "g" "a" "t" "g" "g" "g" "g" "t" "t" "a" "a" "c" "t" "c" "a" "c"
## hidden_state "2" "1" "1" "1" "1" "1" "1" "1" "2" "2" "2" "2" "2" "2" "2" "2"
##              177 178 179 180 181 182 183 184 185 186 187 188 189 190 191 192
## observation  "c" "a" "a" "t" "t" "a" "a" "t" "t" "t" "g" "g" "a" "t" "a" "c"
## hidden_state "2" "2" "1" "1" "2" "2" "2" "1" "1" "1" "1" "2" "2" "2" "2" "2"
##              193 194 195 196 197 198 199 200
## observation  "g" "c" "g" "t" "c" "a" "g" "a"
## hidden_state "2" "1" "1" "1" "1" "1" "2" "2"
\end{verbatim}

\begin{Shaded}
\begin{Highlighting}[]
\CommentTok{#part B)}

\NormalTok{viterbi <-}\StringTok{ }\ControlFlowTok{function}\NormalTok{(A,E,x) \{}
\NormalTok{ v <-}\StringTok{ }\KeywordTok{matrix}\NormalTok{(}\OtherTok{NA}\NormalTok{, }\DataTypeTok{nr=}\KeywordTok{length}\NormalTok{(x), }\DataTypeTok{nc=}\KeywordTok{dim}\NormalTok{(A)[}\DecValTok{1}\NormalTok{])}
\NormalTok{ v[}\DecValTok{1}\NormalTok{,] <-}\StringTok{ }\DecValTok{0}\NormalTok{; v[}\DecValTok{1}\NormalTok{,}\DecValTok{1}\NormalTok{] <-}\StringTok{ }\DecValTok{1}
 \ControlFlowTok{for}\NormalTok{(i }\ControlFlowTok{in} \DecValTok{2}\OperatorTok{:}\KeywordTok{length}\NormalTok{(x)) \{}
 \ControlFlowTok{for}\NormalTok{ (l }\ControlFlowTok{in} \DecValTok{1}\OperatorTok{:}\KeywordTok{dim}\NormalTok{(A)[}\DecValTok{1}\NormalTok{]) \{v[i,l] <-}\StringTok{ }\NormalTok{E[l,x[i]] }\OperatorTok{*}\StringTok{ }\KeywordTok{max}\NormalTok{(v[(i}\DecValTok{-1}\NormalTok{),] }\OperatorTok{*}\StringTok{ }\NormalTok{A[l,])\}}
\NormalTok{ \}}
 \KeywordTok{return}\NormalTok{(v)}
\NormalTok{\}}



\CommentTok{#vit <- viterbi(A,E,dat[1])}
\CommentTok{#vitrowmax <- apply(vit, 1, function(x) which.max(x)) #tracing back max prob path}
\CommentTok{#hiddenstate <- dat[,2]}
\CommentTok{#table(hiddenstate, vitrowmax)}
\CommentTok{#datt <- cbind(dat,vitrowmax)}
\CommentTok{#colnames(datt) <- c("observation","hidden_state","predicted state")}
\CommentTok{#t(datt)}
\end{Highlighting}
\end{Shaded}

\#4. Load the data ``mtcars'' using the command data(mtcars). {[} You
may need to install and \#load the package ``MASS''{]}. Here is a
description of the variables
(\url{https://stat.ethz.ch/Rmanual/R-devel/library/datasets/html/mtcars.html}).
For this data answer the following \#questions using appropriate
correlation tests/confidence intervals. \#(a) Is there a significant
association between the number of gears and the number of \#carburetors?
Test using 𝛼 = 0.05. \#(b) Is there a significant association between
the number of mpg and horsepower? Test \#using 𝛼 = 0.05. \#(c) Test for
a significant association between mpg and number of gears at 𝛼 = 0.05

\begin{Shaded}
\begin{Highlighting}[]
\NormalTok{g<-}\StringTok{ }\NormalTok{mtcars}\OperatorTok{$}\NormalTok{gear}
\NormalTok{c<-}\StringTok{ }\NormalTok{mtcars}\OperatorTok{$}\NormalTok{carb}
\NormalTok{phat<-}\StringTok{ }\KeywordTok{table}\NormalTok{(g, c)}\OperatorTok{/}\KeywordTok{sum}\NormalTok{(}\KeywordTok{length}\NormalTok{(g)}\OperatorTok{+}\StringTok{ }\KeywordTok{length}\NormalTok{(c))}
\NormalTok{phat}
\end{Highlighting}
\end{Shaded}

\begin{verbatim}
##    c
## g          1        2        3        4        6        8
##   3 0.046875 0.062500 0.046875 0.078125 0.000000 0.000000
##   4 0.062500 0.062500 0.000000 0.062500 0.000000 0.000000
##   5 0.000000 0.031250 0.000000 0.015625 0.015625 0.015625
\end{verbatim}

\begin{Shaded}
\begin{Highlighting}[]
\NormalTok{phats<-}\KeywordTok{table}\NormalTok{(g,c)[}\DecValTok{4}\NormalTok{]}
\KeywordTok{col}\NormalTok{(phat)}
\end{Highlighting}
\end{Shaded}

\begin{verbatim}
##      [,1] [,2] [,3] [,4] [,5] [,6]
## [1,]    1    2    3    4    5    6
## [2,]    1    2    3    4    5    6
## [3,]    1    2    3    4    5    6
\end{verbatim}

\begin{Shaded}
\begin{Highlighting}[]
\CommentTok{#install.packages = ("MASS")}
\KeywordTok{data}\NormalTok{(mtcars)}
\KeywordTok{head}\NormalTok{(mtcars)}
\end{Highlighting}
\end{Shaded}

\begin{verbatim}
##                    mpg cyl disp  hp drat    wt  qsec vs am gear carb
## Mazda RX4         21.0   6  160 110 3.90 2.620 16.46  0  1    4    4
## Mazda RX4 Wag     21.0   6  160 110 3.90 2.875 17.02  0  1    4    4
## Datsun 710        22.8   4  108  93 3.85 2.320 18.61  1  1    4    1
## Hornet 4 Drive    21.4   6  258 110 3.08 3.215 19.44  1  0    3    1
## Hornet Sportabout 18.7   8  360 175 3.15 3.440 17.02  0  0    3    2
## Valiant           18.1   6  225 105 2.76 3.460 20.22  1  0    3    1
\end{verbatim}

\begin{Shaded}
\begin{Highlighting}[]
\CommentTok{#Part A all correlation tests}


\KeywordTok{cor}\NormalTok{(g,c)}
\end{Highlighting}
\end{Shaded}

\begin{verbatim}
## [1] 0.2740728
\end{verbatim}

\begin{Shaded}
\begin{Highlighting}[]
\KeywordTok{cor.test}\NormalTok{(g, c)}
\end{Highlighting}
\end{Shaded}

\begin{verbatim}
## 
##  Pearson's product-moment correlation
## 
## data:  g and c
## t = 1.5609, df = 30, p-value = 0.129
## alternative hypothesis: true correlation is not equal to 0
## 95 percent confidence interval:
##  -0.08250603  0.56844218
## sample estimates:
##       cor 
## 0.2740728
\end{verbatim}

\begin{Shaded}
\begin{Highlighting}[]
\CommentTok{#cor.test(g, c, method = "spearman", alternative = "less")}
\CommentTok{#cor.test(g, c, method = "kendall")}
 

\CommentTok{#Part B mpg vs hp}
\NormalTok{m<-}\StringTok{ }\NormalTok{mtcars}\OperatorTok{$}\NormalTok{mpg}
\NormalTok{h<-}\StringTok{ }\NormalTok{mtcars}\OperatorTok{$}\NormalTok{hp}

\KeywordTok{cor}\NormalTok{(m,h)}
\end{Highlighting}
\end{Shaded}

\begin{verbatim}
## [1] -0.7761684
\end{verbatim}

\begin{Shaded}
\begin{Highlighting}[]
\KeywordTok{cor.test}\NormalTok{(m, h)}
\end{Highlighting}
\end{Shaded}

\begin{verbatim}
## 
##  Pearson's product-moment correlation
## 
## data:  m and h
## t = -6.7424, df = 30, p-value = 1.788e-07
## alternative hypothesis: true correlation is not equal to 0
## 95 percent confidence interval:
##  -0.8852686 -0.5860994
## sample estimates:
##        cor 
## -0.7761684
\end{verbatim}

\begin{Shaded}
\begin{Highlighting}[]
\CommentTok{#cor.test(m, h, method = "spearman", alternative = "less")}
\CommentTok{#cor.test(m, h, method = "kendall")}

\CommentTok{#Part C gears vs mpg}
\NormalTok{e<-}\StringTok{ }\NormalTok{mtcars}\OperatorTok{$}\NormalTok{gear}
\NormalTok{p<-}\StringTok{ }\NormalTok{mtcars}\OperatorTok{$}\NormalTok{mpg}

\KeywordTok{cor}\NormalTok{(e,p)}
\end{Highlighting}
\end{Shaded}

\begin{verbatim}
## [1] 0.4802848
\end{verbatim}

\begin{Shaded}
\begin{Highlighting}[]
\KeywordTok{cor.test}\NormalTok{(e, p)}
\end{Highlighting}
\end{Shaded}

\begin{verbatim}
## 
##  Pearson's product-moment correlation
## 
## data:  e and p
## t = 2.9992, df = 30, p-value = 0.005401
## alternative hypothesis: true correlation is not equal to 0
## 95 percent confidence interval:
##  0.1580618 0.7100628
## sample estimates:
##       cor 
## 0.4802848
\end{verbatim}

\begin{Shaded}
\begin{Highlighting}[]
\CommentTok{#cor.test(e, p, method = "spearman", alternative = "less")}
\CommentTok{#cor.test(e, p, method = "kendall")}
\end{Highlighting}
\end{Shaded}

\end{document}
