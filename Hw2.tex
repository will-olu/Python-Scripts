% Options for packages loaded elsewhere
\PassOptionsToPackage{unicode}{hyperref}
\PassOptionsToPackage{hyphens}{url}
%
\documentclass[
]{article}
\usepackage{lmodern}
\usepackage{amssymb,amsmath}
\usepackage{ifxetex,ifluatex}
\ifnum 0\ifxetex 1\fi\ifluatex 1\fi=0 % if pdftex
  \usepackage[T1]{fontenc}
  \usepackage[utf8]{inputenc}
  \usepackage{textcomp} % provide euro and other symbols
\else % if luatex or xetex
  \usepackage{unicode-math}
  \defaultfontfeatures{Scale=MatchLowercase}
  \defaultfontfeatures[\rmfamily]{Ligatures=TeX,Scale=1}
\fi
% Use upquote if available, for straight quotes in verbatim environments
\IfFileExists{upquote.sty}{\usepackage{upquote}}{}
\IfFileExists{microtype.sty}{% use microtype if available
  \usepackage[]{microtype}
  \UseMicrotypeSet[protrusion]{basicmath} % disable protrusion for tt fonts
}{}
\makeatletter
\@ifundefined{KOMAClassName}{% if non-KOMA class
  \IfFileExists{parskip.sty}{%
    \usepackage{parskip}
  }{% else
    \setlength{\parindent}{0pt}
    \setlength{\parskip}{6pt plus 2pt minus 1pt}}
}{% if KOMA class
  \KOMAoptions{parskip=half}}
\makeatother
\usepackage{xcolor}
\IfFileExists{xurl.sty}{\usepackage{xurl}}{} % add URL line breaks if available
\IfFileExists{bookmark.sty}{\usepackage{bookmark}}{\usepackage{hyperref}}
\hypersetup{
  pdftitle={Homework \#2},
  pdfauthor={Ade Olu-Ajeigbe},
  hidelinks,
  pdfcreator={LaTeX via pandoc}}
\urlstyle{same} % disable monospaced font for URLs
\usepackage[margin=1in]{geometry}
\usepackage{color}
\usepackage{fancyvrb}
\newcommand{\VerbBar}{|}
\newcommand{\VERB}{\Verb[commandchars=\\\{\}]}
\DefineVerbatimEnvironment{Highlighting}{Verbatim}{commandchars=\\\{\}}
% Add ',fontsize=\small' for more characters per line
\usepackage{framed}
\definecolor{shadecolor}{RGB}{248,248,248}
\newenvironment{Shaded}{\begin{snugshade}}{\end{snugshade}}
\newcommand{\AlertTok}[1]{\textcolor[rgb]{0.94,0.16,0.16}{#1}}
\newcommand{\AnnotationTok}[1]{\textcolor[rgb]{0.56,0.35,0.01}{\textbf{\textit{#1}}}}
\newcommand{\AttributeTok}[1]{\textcolor[rgb]{0.77,0.63,0.00}{#1}}
\newcommand{\BaseNTok}[1]{\textcolor[rgb]{0.00,0.00,0.81}{#1}}
\newcommand{\BuiltInTok}[1]{#1}
\newcommand{\CharTok}[1]{\textcolor[rgb]{0.31,0.60,0.02}{#1}}
\newcommand{\CommentTok}[1]{\textcolor[rgb]{0.56,0.35,0.01}{\textit{#1}}}
\newcommand{\CommentVarTok}[1]{\textcolor[rgb]{0.56,0.35,0.01}{\textbf{\textit{#1}}}}
\newcommand{\ConstantTok}[1]{\textcolor[rgb]{0.00,0.00,0.00}{#1}}
\newcommand{\ControlFlowTok}[1]{\textcolor[rgb]{0.13,0.29,0.53}{\textbf{#1}}}
\newcommand{\DataTypeTok}[1]{\textcolor[rgb]{0.13,0.29,0.53}{#1}}
\newcommand{\DecValTok}[1]{\textcolor[rgb]{0.00,0.00,0.81}{#1}}
\newcommand{\DocumentationTok}[1]{\textcolor[rgb]{0.56,0.35,0.01}{\textbf{\textit{#1}}}}
\newcommand{\ErrorTok}[1]{\textcolor[rgb]{0.64,0.00,0.00}{\textbf{#1}}}
\newcommand{\ExtensionTok}[1]{#1}
\newcommand{\FloatTok}[1]{\textcolor[rgb]{0.00,0.00,0.81}{#1}}
\newcommand{\FunctionTok}[1]{\textcolor[rgb]{0.00,0.00,0.00}{#1}}
\newcommand{\ImportTok}[1]{#1}
\newcommand{\InformationTok}[1]{\textcolor[rgb]{0.56,0.35,0.01}{\textbf{\textit{#1}}}}
\newcommand{\KeywordTok}[1]{\textcolor[rgb]{0.13,0.29,0.53}{\textbf{#1}}}
\newcommand{\NormalTok}[1]{#1}
\newcommand{\OperatorTok}[1]{\textcolor[rgb]{0.81,0.36,0.00}{\textbf{#1}}}
\newcommand{\OtherTok}[1]{\textcolor[rgb]{0.56,0.35,0.01}{#1}}
\newcommand{\PreprocessorTok}[1]{\textcolor[rgb]{0.56,0.35,0.01}{\textit{#1}}}
\newcommand{\RegionMarkerTok}[1]{#1}
\newcommand{\SpecialCharTok}[1]{\textcolor[rgb]{0.00,0.00,0.00}{#1}}
\newcommand{\SpecialStringTok}[1]{\textcolor[rgb]{0.31,0.60,0.02}{#1}}
\newcommand{\StringTok}[1]{\textcolor[rgb]{0.31,0.60,0.02}{#1}}
\newcommand{\VariableTok}[1]{\textcolor[rgb]{0.00,0.00,0.00}{#1}}
\newcommand{\VerbatimStringTok}[1]{\textcolor[rgb]{0.31,0.60,0.02}{#1}}
\newcommand{\WarningTok}[1]{\textcolor[rgb]{0.56,0.35,0.01}{\textbf{\textit{#1}}}}
\usepackage{graphicx,grffile}
\makeatletter
\def\maxwidth{\ifdim\Gin@nat@width>\linewidth\linewidth\else\Gin@nat@width\fi}
\def\maxheight{\ifdim\Gin@nat@height>\textheight\textheight\else\Gin@nat@height\fi}
\makeatother
% Scale images if necessary, so that they will not overflow the page
% margins by default, and it is still possible to overwrite the defaults
% using explicit options in \includegraphics[width, height, ...]{}
\setkeys{Gin}{width=\maxwidth,height=\maxheight,keepaspectratio}
% Set default figure placement to htbp
\makeatletter
\def\fps@figure{htbp}
\makeatother
\setlength{\emergencystretch}{3em} % prevent overfull lines
\providecommand{\tightlist}{%
  \setlength{\itemsep}{0pt}\setlength{\parskip}{0pt}}
\setcounter{secnumdepth}{-\maxdimen} % remove section numbering

\title{Homework \#2}
\author{Ade Olu-Ajeigbe}
\date{3/3/2021}

\begin{document}
\maketitle

\begin{Shaded}
\begin{Highlighting}[]
\CommentTok{# Use bayes Rule}
\CommentTok{#P(H) = 0.25 #total population with hypertension}
\CommentTok{#P(A | H ) = 0.90 #}
\CommentTok{#P(A | nC ) = 0.05 #}
\CommentTok{#P(A | C) = 0.095}

\CommentTok{#BayesRule -> P(H) *P(A| H)/ P(H) * P(A| H) + P(notH) * P(nH)*P(A | nC)}

\NormalTok{BayesRule <-}\StringTok{ }\NormalTok{(}\FloatTok{0.25}\OperatorTok{*}\StringTok{ }\FloatTok{0.90}\NormalTok{)}\OperatorTok{/}\StringTok{ }\NormalTok{((}\FloatTok{0.25}\OperatorTok{*}\FloatTok{0.90}\NormalTok{) }\OperatorTok{+}\StringTok{ }\NormalTok{(}\FloatTok{0.75}\OperatorTok{*}\FloatTok{0.05}\NormalTok{))}
\NormalTok{BayesRule}
\end{Highlighting}
\end{Shaded}

\begin{verbatim}
## [1] 0.8571429
\end{verbatim}

\hypertarget{including-plots}{%
\subsection{Including Plots}\label{including-plots}}

\begin{enumerate}
\def\labelenumi{(\alph{enumi})}
\tightlist
\item
  Using the first 1000 base pairs of the E,coli data, estimate the
  transition matrix, 𝑃, for a Markov chain. Calculate 𝑃 2, 𝑃4, 𝑃8, 𝑃16.
  (Remember to use \%*\% to do matrix multiplication in R!) Do the
  entries seem to be approaching any specific quantity?
\item
  Assuming 𝜋 = (0.1; 0.15; 0.05; 0.7) and using P from part (a),
  simulate 𝑛 = 3000 observations from this Markov chain. (components of
  𝜋 correspond to A, C, G, T in this order)
\end{enumerate}

\begin{Shaded}
\begin{Highlighting}[]
\CommentTok{#install.packages("seqinr")}
\CommentTok{#install.packages("ade4")}
\CommentTok{#library(ade4)}
\CommentTok{#library(seqinr)}

\CommentTok{#Read in the data in Fasta format}


\CommentTok{#gene <- ("C:/Users/willi/OneDrive/Documents/E.Coli/AE005174v2.fas",package = "seqinr") }
\CommentTok{#x<-c read.fasta (file = gene, as.string = TRUE, seqtype = "AA") [[1]]}

\CommentTok{#str(x)}
\CommentTok{#This is the DNA sequence of ecoli!}
\CommentTok{#ecoli<-c(x[[1]],x[[2]]) #Now you have loaded the DNA seq of Ecoli}



\CommentTok{#some more statistics}
\CommentTok{#str(ecoli)}

\CommentTok{#proportion of C+G in this sequence}
\CommentTok{#sum(table(ecoli)[c("c","g")])/}
 \CommentTok{# sum(table(ecoli)[c("c","g","a","t")])}

\CommentTok{#ecoli table}
\CommentTok{#tab<-table(ecoli)[c("a","c","t","g")]}
\CommentTok{#tab}
\CommentTok{#}
\CommentTok{#ecoli percentages}
\CommentTok{#p<-tab[c("a","c","t","g")]/sum(tab[c("a","c","t","g")])}
\CommentTok{#p}

\CommentTok{#P <- matrix(c(.2, .4, .8, .16))}
\CommentTok{#lengthFirst <- numStrains <- rep(0,nsim)}
\CommentTok{#tic()}
\CommentTok{#for (q in 1:nsim)}
\CommentTok{#\{}
  \CommentTok{#Create the chain object}
 \CommentTok{# chain <- rep(NA, lengthOfChain)}
  
  \CommentTok{#Initialize chain}
  \CommentTok{#chain[1]<-sample(states,1,p=pi)}
  
  \CommentTok{#for (i in 1:(lengthOfChain-1))\{}
    \CommentTok{#chain[i+1]<-sample(states,1,p=P[chain[i],])}
  \CommentTok{#\}}
  
  \CommentTok{#calculate number of strains}
  \CommentTok{#numStrains[q] <- length(unique(chain))}
  
  \CommentTok{#Calculate length before switching. First swith}
 \CommentTok{# lengthFirst[q] <- min(which(diff(chain)!=0))}
\CommentTok{#\}}
\end{Highlighting}
\end{Shaded}

\begin{enumerate}
\def\labelenumi{(\alph{enumi})}
\tightlist
\item
  Find the distribution of two-words in the chain that you simulated in
  2(b). (Get 4 by 4 table with two word frequencies or joint
  probabilities).
\item
  Define a random variable X as the frequency of the nucleotide `C' in a
  chain. Simulate the Markov chain in 2(b) 10000 times and generate a
  histogram of 𝑋. Comment on the distribution of this random variable 𝑋.
\end{enumerate}

\begin{Shaded}
\begin{Highlighting}[]
\CommentTok{#creating table 4 x 4}
\CommentTok{#table }
\end{Highlighting}
\end{Shaded}

\begin{Shaded}
\begin{Highlighting}[]
\CommentTok{#X~Bin(n=25, p=0.1)}

\CommentTok{#P(X > 15)}

\NormalTok{BinDistribution <-}\StringTok{ }\KeywordTok{dbinom}\NormalTok{(}\DecValTok{15}\NormalTok{,}\DecValTok{25}\NormalTok{,}\FloatTok{0.1}\NormalTok{)}
\NormalTok{BinDistribution }\CommentTok{#1.139746e-09}
\end{Highlighting}
\end{Shaded}

\begin{verbatim}
## [1] 1.139746e-09
\end{verbatim}

\begin{Shaded}
\begin{Highlighting}[]
\CommentTok{#Poisson approximation}
\CommentTok{# lamda = n*p = 25 * 0.1}
\NormalTok{lopois <-}\StringTok{ }\KeywordTok{dpois}\NormalTok{(}\DecValTok{15}\NormalTok{, }\FloatTok{2.5}\NormalTok{)}
\NormalTok{lopois }\CommentTok{#5.846074e-08}
\end{Highlighting}
\end{Shaded}

\begin{verbatim}
## [1] 5.846074e-08
\end{verbatim}

\end{document}
