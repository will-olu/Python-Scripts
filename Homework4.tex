% Options for packages loaded elsewhere
\PassOptionsToPackage{unicode}{hyperref}
\PassOptionsToPackage{hyphens}{url}
%
\documentclass[
]{article}
\usepackage{lmodern}
\usepackage{amssymb,amsmath}
\usepackage{ifxetex,ifluatex}
\ifnum 0\ifxetex 1\fi\ifluatex 1\fi=0 % if pdftex
  \usepackage[T1]{fontenc}
  \usepackage[utf8]{inputenc}
  \usepackage{textcomp} % provide euro and other symbols
\else % if luatex or xetex
  \usepackage{unicode-math}
  \defaultfontfeatures{Scale=MatchLowercase}
  \defaultfontfeatures[\rmfamily]{Ligatures=TeX,Scale=1}
\fi
% Use upquote if available, for straight quotes in verbatim environments
\IfFileExists{upquote.sty}{\usepackage{upquote}}{}
\IfFileExists{microtype.sty}{% use microtype if available
  \usepackage[]{microtype}
  \UseMicrotypeSet[protrusion]{basicmath} % disable protrusion for tt fonts
}{}
\makeatletter
\@ifundefined{KOMAClassName}{% if non-KOMA class
  \IfFileExists{parskip.sty}{%
    \usepackage{parskip}
  }{% else
    \setlength{\parindent}{0pt}
    \setlength{\parskip}{6pt plus 2pt minus 1pt}}
}{% if KOMA class
  \KOMAoptions{parskip=half}}
\makeatother
\usepackage{xcolor}
\IfFileExists{xurl.sty}{\usepackage{xurl}}{} % add URL line breaks if available
\IfFileExists{bookmark.sty}{\usepackage{bookmark}}{\usepackage{hyperref}}
\hypersetup{
  pdftitle={Homework \#4},
  pdfauthor={Ade Olu-Ajeigbe},
  hidelinks,
  pdfcreator={LaTeX via pandoc}}
\urlstyle{same} % disable monospaced font for URLs
\usepackage[margin=1in]{geometry}
\usepackage{color}
\usepackage{fancyvrb}
\newcommand{\VerbBar}{|}
\newcommand{\VERB}{\Verb[commandchars=\\\{\}]}
\DefineVerbatimEnvironment{Highlighting}{Verbatim}{commandchars=\\\{\}}
% Add ',fontsize=\small' for more characters per line
\usepackage{framed}
\definecolor{shadecolor}{RGB}{248,248,248}
\newenvironment{Shaded}{\begin{snugshade}}{\end{snugshade}}
\newcommand{\AlertTok}[1]{\textcolor[rgb]{0.94,0.16,0.16}{#1}}
\newcommand{\AnnotationTok}[1]{\textcolor[rgb]{0.56,0.35,0.01}{\textbf{\textit{#1}}}}
\newcommand{\AttributeTok}[1]{\textcolor[rgb]{0.77,0.63,0.00}{#1}}
\newcommand{\BaseNTok}[1]{\textcolor[rgb]{0.00,0.00,0.81}{#1}}
\newcommand{\BuiltInTok}[1]{#1}
\newcommand{\CharTok}[1]{\textcolor[rgb]{0.31,0.60,0.02}{#1}}
\newcommand{\CommentTok}[1]{\textcolor[rgb]{0.56,0.35,0.01}{\textit{#1}}}
\newcommand{\CommentVarTok}[1]{\textcolor[rgb]{0.56,0.35,0.01}{\textbf{\textit{#1}}}}
\newcommand{\ConstantTok}[1]{\textcolor[rgb]{0.00,0.00,0.00}{#1}}
\newcommand{\ControlFlowTok}[1]{\textcolor[rgb]{0.13,0.29,0.53}{\textbf{#1}}}
\newcommand{\DataTypeTok}[1]{\textcolor[rgb]{0.13,0.29,0.53}{#1}}
\newcommand{\DecValTok}[1]{\textcolor[rgb]{0.00,0.00,0.81}{#1}}
\newcommand{\DocumentationTok}[1]{\textcolor[rgb]{0.56,0.35,0.01}{\textbf{\textit{#1}}}}
\newcommand{\ErrorTok}[1]{\textcolor[rgb]{0.64,0.00,0.00}{\textbf{#1}}}
\newcommand{\ExtensionTok}[1]{#1}
\newcommand{\FloatTok}[1]{\textcolor[rgb]{0.00,0.00,0.81}{#1}}
\newcommand{\FunctionTok}[1]{\textcolor[rgb]{0.00,0.00,0.00}{#1}}
\newcommand{\ImportTok}[1]{#1}
\newcommand{\InformationTok}[1]{\textcolor[rgb]{0.56,0.35,0.01}{\textbf{\textit{#1}}}}
\newcommand{\KeywordTok}[1]{\textcolor[rgb]{0.13,0.29,0.53}{\textbf{#1}}}
\newcommand{\NormalTok}[1]{#1}
\newcommand{\OperatorTok}[1]{\textcolor[rgb]{0.81,0.36,0.00}{\textbf{#1}}}
\newcommand{\OtherTok}[1]{\textcolor[rgb]{0.56,0.35,0.01}{#1}}
\newcommand{\PreprocessorTok}[1]{\textcolor[rgb]{0.56,0.35,0.01}{\textit{#1}}}
\newcommand{\RegionMarkerTok}[1]{#1}
\newcommand{\SpecialCharTok}[1]{\textcolor[rgb]{0.00,0.00,0.00}{#1}}
\newcommand{\SpecialStringTok}[1]{\textcolor[rgb]{0.31,0.60,0.02}{#1}}
\newcommand{\StringTok}[1]{\textcolor[rgb]{0.31,0.60,0.02}{#1}}
\newcommand{\VariableTok}[1]{\textcolor[rgb]{0.00,0.00,0.00}{#1}}
\newcommand{\VerbatimStringTok}[1]{\textcolor[rgb]{0.31,0.60,0.02}{#1}}
\newcommand{\WarningTok}[1]{\textcolor[rgb]{0.56,0.35,0.01}{\textbf{\textit{#1}}}}
\usepackage{graphicx,grffile}
\makeatletter
\def\maxwidth{\ifdim\Gin@nat@width>\linewidth\linewidth\else\Gin@nat@width\fi}
\def\maxheight{\ifdim\Gin@nat@height>\textheight\textheight\else\Gin@nat@height\fi}
\makeatother
% Scale images if necessary, so that they will not overflow the page
% margins by default, and it is still possible to overwrite the defaults
% using explicit options in \includegraphics[width, height, ...]{}
\setkeys{Gin}{width=\maxwidth,height=\maxheight,keepaspectratio}
% Set default figure placement to htbp
\makeatletter
\def\fps@figure{htbp}
\makeatother
\setlength{\emergencystretch}{3em} % prevent overfull lines
\providecommand{\tightlist}{%
  \setlength{\itemsep}{0pt}\setlength{\parskip}{0pt}}
\setcounter{secnumdepth}{-\maxdimen} % remove section numbering

\title{Homework \#4}
\author{Ade Olu-Ajeigbe}
\date{4/19/2021}

\begin{document}
\maketitle

1.(a). The first column denotes the ID of the patients and needs to be
dropped. Treat the class variable (redefined as malignant) as your
response and fit a logistic regression model using all 9 covariates and
699 samples to predict class of cancer for patients. Interpret the
fitted model. 1.(b). Use appropriate methods to choose significant
covariates (from all 9) that predict the class of cancer for the
patients. Interpret the fitted model. (Note: Use all 699 observations).

\begin{Shaded}
\begin{Highlighting}[]
\CommentTok{#install.packages("mlbench")}
\KeywordTok{library}\NormalTok{(mlbench)}
\end{Highlighting}
\end{Shaded}

\begin{verbatim}
## Warning: package 'mlbench' was built under R version 4.0.5
\end{verbatim}

\begin{Shaded}
\begin{Highlighting}[]
\KeywordTok{data}\NormalTok{(BreastCancer)}
\NormalTok{data<-BreastCancer  }\CommentTok{#assigning name}


\NormalTok{data}\OperatorTok{$}\NormalTok{Id<-}\OtherTok{NULL}   \CommentTok{#dropping ID}
\NormalTok{data}\OperatorTok{$}\NormalTok{malignant <-}\StringTok{ }\NormalTok{data}\OperatorTok{$}\NormalTok{Class }\OperatorTok{==}\StringTok{ "malignant"}
\NormalTok{data}\OperatorTok{$}\NormalTok{Class<-}\OtherTok{NULL}   \CommentTok{#dropping class variable, now redefined as malignant}
\CommentTok{#All the covariates are factors. They need to be converted to numeric vectors.}
\NormalTok{cols<-}\KeywordTok{c}\NormalTok{(}\DecValTok{1}\OperatorTok{:}\DecValTok{9}\NormalTok{) }\CommentTok{#columns numbers 1 through 9}
\NormalTok{data[cols]<-}\KeywordTok{sapply}\NormalTok{(data[cols], as.numeric)}\CommentTok{#changes first 9 columns to numeric}




\CommentTok{#Missing values}
\KeywordTok{which}\NormalTok{(}\KeywordTok{is.na}\NormalTok{(data}\OperatorTok{$}\NormalTok{Bare.nuclei))}
\end{Highlighting}
\end{Shaded}

\begin{verbatim}
##  [1]  24  41 140 146 159 165 236 250 276 293 295 298 316 322 412 618
\end{verbatim}

\begin{Shaded}
\begin{Highlighting}[]
\CommentTok{#comparision of all covariants }
\NormalTok{model <-}\StringTok{ }\KeywordTok{glm}\NormalTok{(malignant }\OperatorTok{~}\StringTok{ }\NormalTok{.,}\DataTypeTok{family =} \KeywordTok{binomial}\NormalTok{(logit), }\DataTypeTok{data=}\NormalTok{data)}
\KeywordTok{summary}\NormalTok{(model)}
\end{Highlighting}
\end{Shaded}

\begin{verbatim}
## 
## Call:
## glm(formula = malignant ~ ., family = binomial(logit), data = data)
## 
## Deviance Residuals: 
##     Min       1Q   Median       3Q      Max  
## -3.4855  -0.1152  -0.0619   0.0222   2.4702  
## 
## Coefficients:
##                   Estimate Std. Error z value Pr(>|z|)    
## (Intercept)     -10.110096   1.173774  -8.613  < 2e-16 ***
## Cl.thickness      0.535256   0.141938   3.771 0.000163 ***
## Cell.size        -0.005943   0.209158  -0.028 0.977332    
## Cell.shape        0.322136   0.230644   1.397 0.162510    
## Marg.adhesion     0.330694   0.123462   2.679 0.007395 ** 
## Epith.c.size      0.096797   0.156568   0.618 0.536415    
## Bare.nuclei       0.383015   0.093865   4.080 4.49e-05 ***
## Bl.cromatin       0.447401   0.171392   2.610 0.009044 ** 
## Normal.nucleoli   0.213074   0.112894   1.887 0.059109 .  
## Mitoses           0.538551   0.325615   1.654 0.098138 .  
## ---
## Signif. codes:  0 '***' 0.001 '**' 0.01 '*' 0.05 '.' 0.1 ' ' 1
## 
## (Dispersion parameter for binomial family taken to be 1)
## 
##     Null deviance: 884.35  on 682  degrees of freedom
## Residual deviance: 102.90  on 673  degrees of freedom
##   (16 observations deleted due to missingness)
## AIC: 122.9
## 
## Number of Fisher Scoring iterations: 8
\end{verbatim}

\begin{Shaded}
\begin{Highlighting}[]
\CommentTok{#model$coefficients}

\CommentTok{#The summary showed that Bare.nuclei, Cl.thickness, Marg.adhesion, and Bl.cromatin are the most significant covariants.}

\CommentTok{#PartB)}


\NormalTok{model2 <-}\StringTok{ }\KeywordTok{glm}\NormalTok{(malignant }\OperatorTok{~}\StringTok{ }\NormalTok{Cl.thickness }\OperatorTok{+}\StringTok{ }\NormalTok{Marg.adhesion }\OperatorTok{+}\StringTok{ }\NormalTok{Bl.cromatin }\OperatorTok{+}\StringTok{ }\NormalTok{Bare.nuclei, }\DataTypeTok{family =} \KeywordTok{binomial}\NormalTok{(logit), }\DataTypeTok{data=}\NormalTok{data)}
\KeywordTok{summary}\NormalTok{(model2)}
\end{Highlighting}
\end{Shaded}

\begin{verbatim}
## 
## Call:
## glm(formula = malignant ~ Cl.thickness + Marg.adhesion + Bl.cromatin + 
##     Bare.nuclei, family = binomial(logit), data = data)
## 
## Deviance Residuals: 
##     Min       1Q   Median       3Q      Max  
## -3.6964  -0.1451  -0.0609   0.0232   2.4476  
## 
## Coefficients:
##                Estimate Std. Error z value Pr(>|z|)    
## (Intercept)   -10.11370    1.03264  -9.794  < 2e-16 ***
## Cl.thickness    0.81166    0.12585   6.450 1.12e-10 ***
## Marg.adhesion   0.43412    0.11403   3.807 0.000141 ***
## Bl.cromatin     0.70154    0.15196   4.616 3.90e-06 ***
## Bare.nuclei     0.48136    0.08816   5.460 4.76e-08 ***
## ---
## Signif. codes:  0 '***' 0.001 '**' 0.01 '*' 0.05 '.' 0.1 ' ' 1
## 
## (Dispersion parameter for binomial family taken to be 1)
## 
##     Null deviance: 884.35  on 682  degrees of freedom
## Residual deviance: 125.77  on 678  degrees of freedom
##   (16 observations deleted due to missingness)
## AIC: 135.77
## 
## Number of Fisher Scoring iterations: 8
\end{verbatim}

\begin{Shaded}
\begin{Highlighting}[]
\CommentTok{#High AIC scores for all comparisons. The group comparison had a high AIC value}
\end{Highlighting}
\end{Shaded}

\begin{enumerate}
\def\labelenumi{\arabic{enumi}.}
\setcounter{enumi}{1}
\tightlist
\item
  (a). Fit the model 1(b) on a training set (75\%) and provide measure
  of accuracy using k-fold cross validation, with k = 10. (b). Repeat
  with leave one out cross validation method and comment on the
  performance of your fitted model from 1(b).
\end{enumerate}

\begin{Shaded}
\begin{Highlighting}[]
\KeywordTok{library}\NormalTok{(caret)}
\end{Highlighting}
\end{Shaded}

\begin{verbatim}
## Warning: package 'caret' was built under R version 4.0.5
\end{verbatim}

\begin{verbatim}
## Loading required package: lattice
\end{verbatim}

\begin{verbatim}
## Loading required package: ggplot2
\end{verbatim}

\begin{verbatim}
## Warning: package 'ggplot2' was built under R version 4.0.3
\end{verbatim}

\begin{Shaded}
\begin{Highlighting}[]
\CommentTok{#a)}
\CommentTok{#creating the k-fold cross validation}

\NormalTok{Train <-}\StringTok{ }\KeywordTok{createDataPartition}\NormalTok{(data}\OperatorTok{$}\NormalTok{malignant, }\DataTypeTok{p=} \FloatTok{0.75}\NormalTok{, }\DataTypeTok{list=} \OtherTok{FALSE}\NormalTok{)}

\NormalTok{training <-}\StringTok{ }\NormalTok{data[ Train, ]}
\NormalTok{testing <-}\StringTok{ }\NormalTok{data[}\OperatorTok{-}\NormalTok{Train, ]}


\CommentTok{#training the control set}
\NormalTok{train.control <-}\StringTok{ }\KeywordTok{trainControl}\NormalTok{(}\DataTypeTok{method =} \StringTok{"repeatedcv"}\NormalTok{, }\DataTypeTok{number =} \DecValTok{10}\NormalTok{, }\DataTypeTok{savePredictions =} \OtherTok{TRUE}\NormalTok{, }\DataTypeTok{repeats =} \DecValTok{5}\NormalTok{ )}

\NormalTok{training}\OperatorTok{$}\NormalTok{malignant<-}\StringTok{ }\KeywordTok{factor}\NormalTok{(training}\OperatorTok{$}\NormalTok{malignant)}

\CommentTok{#new model with the parameters of part b}
\CommentTok{#model5 <- train(malignant ~ Cl.thickness + Marg.adhesion + Bl.cromatin + Bare.nuclei, data = training, trControl = train.control, method = "glm", family = "binomial")}
\CommentTok{#print(model5)}

\CommentTok{#testing$malignant<-factor(testing$malignant)}
\CommentTok{#pred <- predict(model5, newdata=testing)}
\CommentTok{#confusionMatrix(data=pred, testing$malignant)}



\CommentTok{#b)}

\NormalTok{TrainH<-}\StringTok{ }\KeywordTok{createDataPartition}\NormalTok{(data}\OperatorTok{$}\NormalTok{malignant, }\DataTypeTok{p =}\FloatTok{0.75}\NormalTok{, }\DataTypeTok{list =}\OtherTok{FALSE}\NormalTok{)}
\NormalTok{training <-}\StringTok{ }\NormalTok{data[ Train, ]}
\NormalTok{testing <-}\StringTok{ }\NormalTok{data[}\OperatorTok{-}\NormalTok{Train, ]}

\KeywordTok{head}\NormalTok{(data)}
\end{Highlighting}
\end{Shaded}

\begin{verbatim}
##   Cl.thickness Cell.size Cell.shape Marg.adhesion Epith.c.size Bare.nuclei
## 1            5         1          1             1            2           1
## 2            5         4          4             5            7          10
## 3            3         1          1             1            2           2
## 4            6         8          8             1            3           4
## 5            4         1          1             3            2           1
## 6            8        10         10             8            7          10
##   Bl.cromatin Normal.nucleoli Mitoses malignant
## 1           3               1       1     FALSE
## 2           3               2       1     FALSE
## 3           3               1       1     FALSE
## 4           3               7       1     FALSE
## 5           3               1       1     FALSE
## 6           9               7       1      TRUE
\end{verbatim}

\begin{Shaded}
\begin{Highlighting}[]
\NormalTok{model3 <-}\StringTok{ }\KeywordTok{step}\NormalTok{(model2, }\DataTypeTok{data =}\NormalTok{ training)}
\end{Highlighting}
\end{Shaded}

\begin{verbatim}
## Start:  AIC=135.77
## malignant ~ Cl.thickness + Marg.adhesion + Bl.cromatin + Bare.nuclei
## 
##                 Df Deviance    AIC
## <none>               125.78 135.78
## - Marg.adhesion  1   142.92 150.92
## - Bl.cromatin    1   155.44 163.44
## - Bare.nuclei    1   166.89 174.89
## - Cl.thickness   1   197.02 205.02
\end{verbatim}

\begin{Shaded}
\begin{Highlighting}[]
\KeywordTok{summary}\NormalTok{(model3)}
\end{Highlighting}
\end{Shaded}

\begin{verbatim}
## 
## Call:
## glm(formula = malignant ~ Cl.thickness + Marg.adhesion + Bl.cromatin + 
##     Bare.nuclei, family = binomial(logit), data = data)
## 
## Deviance Residuals: 
##     Min       1Q   Median       3Q      Max  
## -3.6964  -0.1451  -0.0609   0.0232   2.4476  
## 
## Coefficients:
##                Estimate Std. Error z value Pr(>|z|)    
## (Intercept)   -10.11370    1.03264  -9.794  < 2e-16 ***
## Cl.thickness    0.81166    0.12585   6.450 1.12e-10 ***
## Marg.adhesion   0.43412    0.11403   3.807 0.000141 ***
## Bl.cromatin     0.70154    0.15196   4.616 3.90e-06 ***
## Bare.nuclei     0.48136    0.08816   5.460 4.76e-08 ***
## ---
## Signif. codes:  0 '***' 0.001 '**' 0.01 '*' 0.05 '.' 0.1 ' ' 1
## 
## (Dispersion parameter for binomial family taken to be 1)
## 
##     Null deviance: 884.35  on 682  degrees of freedom
## Residual deviance: 125.77  on 678  degrees of freedom
##   (16 observations deleted due to missingness)
## AIC: 135.77
## 
## Number of Fisher Scoring iterations: 8
\end{verbatim}

\begin{Shaded}
\begin{Highlighting}[]
\NormalTok{pred<-}\KeywordTok{predict}\NormalTok{(model3, }\DataTypeTok{mewdata =}\NormalTok{testing, }\DataTypeTok{type =} \StringTok{"response"}\NormalTok{)}
\NormalTok{acc <-}\StringTok{ }\OtherTok{NULL}
\NormalTok{results<-}\StringTok{ }\KeywordTok{ifelse}\NormalTok{(pred }\OperatorTok{>}\StringTok{ }\FloatTok{0.5}\NormalTok{, }\DecValTok{1}\NormalTok{, }\DecValTok{0}\NormalTok{)}
\NormalTok{answers <-}\StringTok{ }\NormalTok{testing}\OperatorTok{$}\NormalTok{malignant}
\NormalTok{misClasificError <-}\StringTok{ }\KeywordTok{mean}\NormalTok{(answers }\OperatorTok{!=}\StringTok{ }\NormalTok{results)}
\end{Highlighting}
\end{Shaded}

\begin{verbatim}
## Warning in answers != results: longer object length is not a multiple of shorter
## object length
\end{verbatim}

\begin{Shaded}
\begin{Highlighting}[]
\NormalTok{accuracy <-}\StringTok{ }\KeywordTok{table}\NormalTok{(results, results)}
\KeywordTok{sum}\NormalTok{(}\KeywordTok{diag}\NormalTok{(accuracy))}\OperatorTok{/}\StringTok{ }\KeywordTok{sum}\NormalTok{(accuracy)}
\end{Highlighting}
\end{Shaded}

\begin{verbatim}
## [1] 1
\end{verbatim}

\begin{Shaded}
\begin{Highlighting}[]
\NormalTok{acc1<-}\StringTok{ }\OtherTok{NULL}
\ControlFlowTok{for}\NormalTok{(i }\ControlFlowTok{in} \DecValTok{1}\OperatorTok{:}\KeywordTok{nrow}\NormalTok{(data))}
\NormalTok{\{}
 \CommentTok{# Train-test splitting}
 \CommentTok{# 699 samples -> fitting}
 \CommentTok{# 1 sample -> testing}
\NormalTok{ train <-}\StringTok{ }\NormalTok{data[}\OperatorTok{-}\NormalTok{i,]}
\NormalTok{ test <-}\StringTok{ }\NormalTok{data[i,]}

 \CommentTok{# Fitting}
\NormalTok{ model6 <-}\StringTok{ }\KeywordTok{glm}\NormalTok{(malignant }\OperatorTok{~}\StringTok{ }\NormalTok{Cl.thickness }\OperatorTok{+}\StringTok{ }\NormalTok{Marg.adhesion }\OperatorTok{+}\StringTok{ }\NormalTok{Bl.cromatin }\OperatorTok{+}\StringTok{ }\NormalTok{Bare.nuclei,}
 \DataTypeTok{family =} \KeywordTok{binomial}\NormalTok{(logit), }\DataTypeTok{data =}\NormalTok{ data)}

 \CommentTok{# Predict results}
\NormalTok{ pred1 <-}\StringTok{ }\KeywordTok{predict}\NormalTok{(model6,}\DataTypeTok{newdata=}\NormalTok{test,}\DataTypeTok{type=}\StringTok{"response"}\NormalTok{)}

 \CommentTok{# If prob > 0.5 then 1, else 0}
\NormalTok{ results1 <-}\StringTok{ }\KeywordTok{ifelse}\NormalTok{(pred1 }\OperatorTok{>}\StringTok{ }\FloatTok{0.5}\NormalTok{,}\DecValTok{1}\NormalTok{,}\DecValTok{0}\NormalTok{)}
 
 \CommentTok{# Actual answers}
\NormalTok{ answers1 <-}\StringTok{ }\NormalTok{test}\OperatorTok{$}\NormalTok{malignant}

 \CommentTok{# Calculate accuracy}
\NormalTok{ misClasificError1 <-}\StringTok{ }\KeywordTok{mean}\NormalTok{(answers1 }\OperatorTok{!=}\StringTok{ }\NormalTok{results1)}

 \CommentTok{# Collecting results}
\NormalTok{ acc1[i] <-}\StringTok{ }\DecValTok{1}\OperatorTok{-}\NormalTok{misClasificError1}
\NormalTok{\}}
\KeywordTok{mean}\NormalTok{(acc1)}
\end{Highlighting}
\end{Shaded}

\begin{verbatim}
## [1] NA
\end{verbatim}

\begin{Shaded}
\begin{Highlighting}[]
\KeywordTok{hist}\NormalTok{(acc1,}\DataTypeTok{xlab=}\StringTok{'Accuracy'}\NormalTok{,}\DataTypeTok{ylab=}\StringTok{'Freq'}\NormalTok{,}\DataTypeTok{main=}\StringTok{'Accuracy LOOCV'}\NormalTok{,}
 \DataTypeTok{col=}\StringTok{'cyan'}\NormalTok{,}\DataTypeTok{border=}\StringTok{'blue'}\NormalTok{,}\DataTypeTok{density=}\DecValTok{30}\NormalTok{)}
\end{Highlighting}
\end{Shaded}

\includegraphics{Homework4_files/figure-latex/unnamed-chunk-2-1.pdf}

\begin{enumerate}
\def\labelenumi{\arabic{enumi}.}
\setcounter{enumi}{2}
\tightlist
\item
  Now ignore the class variable and consider the 9 covariates in your
  model. (a). Perform Principal Component Analysis on these 9 variables
  to retain 90\% variation in the data. Interpret the components. (b).
  Plot the PC1versus PC2 and overlap the type of cancer coded by
  different colors. (follow the iris data example in class; plot of PCs
  with flower species). Interpret the plot.
\end{enumerate}

\begin{Shaded}
\begin{Highlighting}[]
\CommentTok{#install.packages("ggfortify")}
\KeywordTok{library}\NormalTok{(ggfortify)}
\end{Highlighting}
\end{Shaded}

\begin{verbatim}
## Warning: package 'ggfortify' was built under R version 4.0.5
\end{verbatim}

\begin{Shaded}
\begin{Highlighting}[]
\CommentTok{#install.packages("mlbench")}
\KeywordTok{library}\NormalTok{(mlbench)}

\NormalTok{data2<-}\StringTok{ }\KeywordTok{data.frame}\NormalTok{(}\KeywordTok{t}\NormalTok{(}\KeywordTok{na.omit}\NormalTok{(}\KeywordTok{t}\NormalTok{(data))))}
\NormalTok{pca <-}\StringTok{ }\KeywordTok{prcomp}\NormalTok{(data2, }\DataTypeTok{center =} \OtherTok{TRUE}\NormalTok{, }\DataTypeTok{scale. =} \OtherTok{TRUE}\NormalTok{)}
\KeywordTok{summary}\NormalTok{(pca)}
\end{Highlighting}
\end{Shaded}

\begin{verbatim}
## Importance of components:
##                           PC1     PC2     PC3     PC4     PC5     PC6    PC7
## Standard deviation     2.4663 0.86852 0.73948 0.64255 0.61312 0.54701 0.5126
## Proportion of Variance 0.6758 0.08381 0.06076 0.04587 0.04177 0.03325 0.0292
## Cumulative Proportion  0.6758 0.75967 0.82043 0.86630 0.90807 0.94131 0.9705
##                            PC8     PC9
## Standard deviation     0.41852 0.30039
## Proportion of Variance 0.01946 0.01003
## Cumulative Proportion  0.98997 1.00000
\end{verbatim}

\begin{Shaded}
\begin{Highlighting}[]
\CommentTok{#B)}

\CommentTok{#taking THE PC1 and PC2 }
\KeywordTok{autoplot}\NormalTok{(pca, }\DataTypeTok{data =}\NormalTok{ data, }\DataTypeTok{loadings =} \OtherTok{TRUE}\NormalTok{, }\DataTypeTok{loading.label =} \OtherTok{TRUE}\NormalTok{, }\DataTypeTok{colour =} \StringTok{'malignant'}\NormalTok{, }\DataTypeTok{frame =} \OtherTok{TRUE}\NormalTok{, }\DataTypeTok{frame.type =} \StringTok{'norm'}\NormalTok{)}
\end{Highlighting}
\end{Shaded}

\begin{verbatim}
## Warning: `select_()` was deprecated in dplyr 0.7.0.
## Please use `select()` instead.
\end{verbatim}

\includegraphics{Homework4_files/figure-latex/unnamed-chunk-3-1.pdf}

\end{document}
